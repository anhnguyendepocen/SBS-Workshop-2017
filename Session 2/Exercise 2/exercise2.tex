

\documentclass[12pt,a4paper]{article}
\usepackage{amsmath}
\usepackage{enumerate}
\usepackage[cm]{fullpage}

\begin{document}
\setlength\parindent{0cm}
\title{\Large{\textbf{Introduction to \texttt{R}}}\\
\textit{Session 2 exercises}}
\author{Statistical Consulting Center}
\date{1 March, 2017}
\maketitle
 

\section{Write your own function}
\label{sec:func}
\begin{enumerate}[(i)]
  \item\label{itm:sefunc} A simple function  calculate the standard
    error of the mean is given in the Session 2 slide, i.e.
%\clearpage\newpage  
  \begin{verbatim}
  mystder <- function(x){
       mysd <- sd(x, na.rm = T)
       n <- length(x)
       mysd/sqrt(n)
  }
  \end{verbatim}
  Copy this code and paste it into your \texttt{R} script, submit it
  to the \texttt{R} console. 
  \item \label{itm:sfunc} Modify the function in
    \ref{sec:func}(\ref{itm:sefunc})  
    so that the output will have only 2 decimal places.
  \item Calculate the standard error of mean \texttt{age} using the
    function you've created in \ref{sec:func}(\ref{itm:sfunc}).
\end{enumerate}


\section{Installing an \texttt{R} package}
\label{sec:pack}
\texttt{R} packages are collections of user-defined functions. The function
\texttt{std.error}, for example, is contained in the \texttt{plotrix} package.
\begin{enumerate}[(i)]
  \item Let's look at what happens when we try to use a function
    before actually installing the package in which it is
    contained. E.g. Calculate the standard error of the mean age using
    \texttt{std.error}.  
\item Install the package \texttt{plotrix} into your \texttt{R}
  session by following the instructions below: 
  \begin{enumerate}
  \item Select \texttt{Packages} from the bottom right panel of your
    Rstudio interface.
  \item Click on \texttt{Install Packages} icon just below \texttt{Packages}.
  \item Type \texttt{plotrix} in the blank space provided below
    ``\texttt{Packages (separate multiple with space or comma):}''
  \item Click on \texttt{No} if it asks you to restart \texttt{R}
  \item Submit the code \texttt{library(plotrix)} to \texttt{R} Console
    to complete this installation.
  \end{enumerate}
\item Now, use \texttt{std.error} to calculate the standard error of
  the mean age. 
\item Try writing your own code to calculate the standard error of the
  mean age. Hint: This only requires one line of code. Use online resources
if you cannot remember how the standard error is calculated.
\end{enumerate}
 
\section{Subsetting datasets}
\label{sec:sub}
\begin{enumerate}[(i)]
\item \label{itm:1way} Produce a one-way frequency table for \texttt{q1a}.
\item \label{itm:na} Find out which participants have chosen ``Can't
  choose'' for this questions.
\item Now produce the frequency table in
  \ref{sec:sub}(\ref{itm:1way}), excluding those participants in
  \ref{sec:sub}(\ref{itm:na}).
\item Calculate the mean age for male participants.
\item Calculate the mean age for male participants who make less than
  \$10000 a year.
\item Calculate the mean age for European male participants who make less than
  \$10000 a year.
\end{enumerate}
 

\section{Challenge}
\label{sec:cha}
%Thue function given in \ref{sec:func} suppose to calculate the
%standard error. Actually, it is only correct if the input does NOT
%contain missing values. This is because the \texttt{length()} function
%will count missing values as well. For example:
%<<>>=
%test <- c(1, 2, 3, 4, NA)
%length(test)
%@
%It returns \Sexpr{length(test)}

Modify the function given in \ref{sec:func}, so that the function will
return a 95\% confidence interval. Hint: A 95\% confidence interval of
a variable \texttt{x} is given by the mean of \texttt{x} $\pm$ 1.96
$\times$ the
standard error of \texttt{x}. You might find the \texttt{paste()}
function useful.
\end{document}
