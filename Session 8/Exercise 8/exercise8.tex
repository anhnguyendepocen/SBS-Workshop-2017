

\documentclass[12pt,a4paper]{article}\usepackage[]{graphicx}\usepackage[]{color}
%% maxwidth is the original width if it is less than linewidth
%% otherwise use linewidth (to make sure the graphics do not exceed the margin)
\makeatletter
\def\maxwidth{ %
  \ifdim\Gin@nat@width>\linewidth
    \linewidth
  \else
    \Gin@nat@width
  \fi
}
\makeatother

\definecolor{fgcolor}{rgb}{0.345, 0.345, 0.345}
\newcommand{\hlnum}[1]{\textcolor[rgb]{0.686,0.059,0.569}{#1}}%
\newcommand{\hlstr}[1]{\textcolor[rgb]{0.192,0.494,0.8}{#1}}%
\newcommand{\hlcom}[1]{\textcolor[rgb]{0.678,0.584,0.686}{\textit{#1}}}%
\newcommand{\hlopt}[1]{\textcolor[rgb]{0,0,0}{#1}}%
\newcommand{\hlstd}[1]{\textcolor[rgb]{0.345,0.345,0.345}{#1}}%
\newcommand{\hlkwa}[1]{\textcolor[rgb]{0.161,0.373,0.58}{\textbf{#1}}}%
\newcommand{\hlkwb}[1]{\textcolor[rgb]{0.69,0.353,0.396}{#1}}%
\newcommand{\hlkwc}[1]{\textcolor[rgb]{0.333,0.667,0.333}{#1}}%
\newcommand{\hlkwd}[1]{\textcolor[rgb]{0.737,0.353,0.396}{\textbf{#1}}}%
\let\hlipl\hlkwb

\usepackage{framed}
\makeatletter
\newenvironment{kframe}{%
 \def\at@end@of@kframe{}%
 \ifinner\ifhmode%
  \def\at@end@of@kframe{\end{minipage}}%
  \begin{minipage}{\columnwidth}%
 \fi\fi%
 \def\FrameCommand##1{\hskip\@totalleftmargin \hskip-\fboxsep
 \colorbox{shadecolor}{##1}\hskip-\fboxsep
     % There is no \\@totalrightmargin, so:
     \hskip-\linewidth \hskip-\@totalleftmargin \hskip\columnwidth}%
 \MakeFramed {\advance\hsize-\width
   \@totalleftmargin\z@ \linewidth\hsize
   \@setminipage}}%
 {\par\unskip\endMakeFramed%
 \at@end@of@kframe}
\makeatother

\definecolor{shadecolor}{rgb}{.97, .97, .97}
\definecolor{messagecolor}{rgb}{0, 0, 0}
\definecolor{warningcolor}{rgb}{1, 0, 1}
\definecolor{errorcolor}{rgb}{1, 0, 0}
\newenvironment{knitrout}{}{} % an empty environment to be redefined in TeX

\usepackage{alltt}
\usepackage{amsmath}
\usepackage{enumerate}
\usepackage[cm]{fullpage}
\usepackage{graphicx}
\IfFileExists{upquote.sty}{\usepackage{upquote}}{}
\begin{document}
\setlength\parindent{0cm}
%\setlength{\oddsidemargin}{0.25cm}
%\setlength{\evensidemargin}{0.25cm}
\title{\Large{\textbf{Introduction to \texttt{R}}}\\
\textit{Session 8 exercises}}
\author{Statistical Consulting Centre}
\date{2 March, 2017}
\maketitle
 
 


\section{Linear regression} 
\label{sec:lm}
\begin{enumerate}[(i)]
\item Perform a linear regression between age (explanatory variable) and nerdy score (dependent variable).

\item Are the estimated intercept and slope significantly different from zero?

\item Examine the residuals of the fitted linear model.

\item Add the fitted line to the scatterplot of nerdy score against age.

\item What conclusions can you draw? Do you think age and nerdy score are linearly correlated?
\end{enumerate}


\section{Logistic Regression} 
\label{sec:log}
\begin{enumerate}[(i)]

\subsection{Continuous explanatory variable}

\item Suppose we want to model the probability of being male, i.e., \texttt{gender = Male}. First, ensure that \texttt{gender} is a variable with a correct type. 

\item Fit a logistic model with \texttt{gender} as the response variable and \texttt{nerdy.sc} as the explanatory variable.

\item Perform an analysis of deviance to determine the overall significance of \texttt{nerdy.sc}.

\item Calculate the estimated slope of the logistic regression. What can you conclude about the slope?

\end{enumerate}

\subsection{Categorical explanatory variable}
\begin{enumerate}[(i)]
\item We now want to model the probability of living with a partner given age group. \texttt{partner} is already of type \texttt{factor}. Now, generate a one-way table of \texttt{partner} to examine its contents. %Notice that there are only \texttt{Yes} and \texttt{No} cases remaining.

\item Set \texttt{partner} = \texttt{Yes} as the reference level.

\item Once again geenerate the one-way frequency table of \texttt{partner}.

\item Fit a logistic model with \texttt{partner} as the response variable and \texttt{age.group} as the explanatory variable.

\item Is \texttt{age.group} a significant predictor of whether or not an individual in particular age group has a partner?

%\item[] Not at the 5\% level of significance since $p$ = myp.
\item Generate a two-way frequency table of \texttt{partner} against \texttt{age.group}. 

\item Convert these frequencies to percentages of age group total. Does this table agree with your earlier conclusion?

%\item[] Yes, since the percentages of \texttt{Yes} and \texttt{No} are approximately the same across age groups.
\end{enumerate}


\end{document}
