

\documentclass[12pt,a4paper]{article}\usepackage[]{graphicx}\usepackage[]{color}
%% maxwidth is the original width if it is less than linewidth
%% otherwise use linewidth (to make sure the graphics do not exceed the margin)
\makeatletter
\def\maxwidth{ %
  \ifdim\Gin@nat@width>\linewidth
    \linewidth
  \else
    \Gin@nat@width
  \fi
}
\makeatother

\definecolor{fgcolor}{rgb}{0.345, 0.345, 0.345}
\newcommand{\hlnum}[1]{\textcolor[rgb]{0.686,0.059,0.569}{#1}}%
\newcommand{\hlstr}[1]{\textcolor[rgb]{0.192,0.494,0.8}{#1}}%
\newcommand{\hlcom}[1]{\textcolor[rgb]{0.678,0.584,0.686}{\textit{#1}}}%
\newcommand{\hlopt}[1]{\textcolor[rgb]{0,0,0}{#1}}%
\newcommand{\hlstd}[1]{\textcolor[rgb]{0.345,0.345,0.345}{#1}}%
\newcommand{\hlkwa}[1]{\textcolor[rgb]{0.161,0.373,0.58}{\textbf{#1}}}%
\newcommand{\hlkwb}[1]{\textcolor[rgb]{0.69,0.353,0.396}{#1}}%
\newcommand{\hlkwc}[1]{\textcolor[rgb]{0.333,0.667,0.333}{#1}}%
\newcommand{\hlkwd}[1]{\textcolor[rgb]{0.737,0.353,0.396}{\textbf{#1}}}%
\let\hlipl\hlkwb

\usepackage{framed}
\makeatletter
\newenvironment{kframe}{%
 \def\at@end@of@kframe{}%
 \ifinner\ifhmode%
  \def\at@end@of@kframe{\end{minipage}}%
  \begin{minipage}{\columnwidth}%
 \fi\fi%
 \def\FrameCommand##1{\hskip\@totalleftmargin \hskip-\fboxsep
 \colorbox{shadecolor}{##1}\hskip-\fboxsep
     % There is no \\@totalrightmargin, so:
     \hskip-\linewidth \hskip-\@totalleftmargin \hskip\columnwidth}%
 \MakeFramed {\advance\hsize-\width
   \@totalleftmargin\z@ \linewidth\hsize
   \@setminipage}}%
 {\par\unskip\endMakeFramed%
 \at@end@of@kframe}
\makeatother

\definecolor{shadecolor}{rgb}{.97, .97, .97}
\definecolor{messagecolor}{rgb}{0, 0, 0}
\definecolor{warningcolor}{rgb}{1, 0, 1}
\definecolor{errorcolor}{rgb}{1, 0, 0}
\newenvironment{knitrout}{}{} % an empty environment to be redefined in TeX

\usepackage{alltt}
\usepackage{amsmath}
\usepackage{enumerate}
\usepackage[cm]{fullpage}
\IfFileExists{upquote.sty}{\usepackage{upquote}}{}
\begin{document}
\setlength\parindent{0cm}
%\setlength{\oddsidemargin}{0.25cm}
%\setlength{\evensidemargin}{0.25cm}
\title{\Large{\textbf{Introduction to \texttt{R}}}\\
\textit{Session 2 exercises}}
\author{Statistical Consulting Center}
\date{1 March, 2017}
\maketitle
 

\section{Write your own function}
\label{sec:func}


\begin{enumerate}[(i)]
  \item\label{itm:sefunc} In Session 2 you were shown a simple function to calculate the standard error of the mean (SEM), i.e.
%\clearpage\newpage  
  \begin{verbatim}
  mystder <- function(x){
       mysd <- sd(x, na.rm = TRUE)
       n <- length(x)
       mysd/sqrt(n)
  }
  \end{verbatim}
 Type the above code into your \texttt{R} script and submit it
  to the \texttt{R} console. 
  \item \label{itm:sfunc} Modify the function in
    \ref{sec:func}(\ref{itm:sefunc})  
    so that the output will have only 2 decimal places.
\begin{knitrout}
\definecolor{shadecolor}{rgb}{0.969, 0.969, 0.969}\color{fgcolor}\begin{kframe}
\begin{alltt}
\hlstd{mystder} \hlkwb{<-} \hlkwa{function}\hlstd{(}\hlkwc{x}\hlstd{)\{}
       \hlstd{mysd} \hlkwb{<-} \hlkwd{sd}\hlstd{(x,} \hlkwc{na.rm} \hlstd{=} \hlnum{TRUE}\hlstd{)}
       \hlstd{n} \hlkwb{<-} \hlkwd{length}\hlstd{(x)}
       \hlkwd{round}\hlstd{(mysd}\hlopt{/}\hlkwd{sqrt}\hlstd{(n),} \hlnum{2}\hlstd{)}
\hlstd{\}}
\end{alltt}
\end{kframe}
\end{knitrout}


  \item Calculate the SEM of \texttt{age} using the
    function you created in \ref{sec:func}(\ref{itm:sfunc}).
\begin{knitrout}
\definecolor{shadecolor}{rgb}{0.969, 0.969, 0.969}\color{fgcolor}\begin{kframe}
\begin{alltt}
\hlkwd{mystder}\hlstd{(sports.df}\hlopt{$}\hlstd{age)}
\end{alltt}
\begin{verbatim}
[1] 0.55
\end{verbatim}
\end{kframe}
\end{knitrout}


\end{enumerate}


\section{Installing an \texttt{R} package}
\label{sec:pack}
\texttt{R} packages are collections of user-defined functions. The function
\texttt{std.error}, for example, is contained in the \texttt{plotrix} package.
\begin{enumerate}[(i)]
  \item Let's look at what happens when we try to use a function
    before actually installing on our computer the package in which it is
    contained. E.g. Calculate the SEM of age using
    \texttt{std.error}.  
\begin{knitrout}
\definecolor{shadecolor}{rgb}{0.969, 0.969, 0.969}\color{fgcolor}\begin{kframe}
\begin{alltt}
\hlkwd{std.error}\hlstd{(sports.df}\hlopt{$}\hlstd{age)}
\end{alltt}


{\ttfamily\noindent\bfseries\color{errorcolor}{Error in eval(expr, envir, enclos): could not find function "{}std.error"{}}}\end{kframe}
\end{knitrout}
\item Install the package \texttt{plotrix} while in your \texttt{R}
  session by following the instructions below: 
  \begin{enumerate}
  \item Select \texttt{Packages} from the bottom right panel of your
    Rstudio interface.
  \item Click on the \texttt{Install Packages} icon just below \texttt{Packages}.
  \item Type \texttt{plotrix} in the blank space provided below
    ``\texttt{Packages (separate multiple with space or comma):}''
  \item Click on \texttt{No} if you are asked you to restart \texttt{R}
  \item Submit the code \texttt{library(plotrix)} to the \texttt{R} console
    to make the functions contained in \texttt{plotrix} available in the current \texttt{R} session.
\begin{knitrout}
\definecolor{shadecolor}{rgb}{0.969, 0.969, 0.969}\color{fgcolor}\begin{kframe}
\begin{alltt}
\hlkwd{library}\hlstd{(plotrix)}
\end{alltt}
\end{kframe}
\end{knitrout}
  \end{enumerate}
\item Now, use \texttt{std.error} to calculate the standard error of
  the mean age. 
\begin{knitrout}
\definecolor{shadecolor}{rgb}{0.969, 0.969, 0.969}\color{fgcolor}\begin{kframe}
\begin{alltt}
\hlkwd{std.error}\hlstd{(sports.df}\hlopt{$}\hlstd{age)}
\end{alltt}
\begin{verbatim}
[1] 0.5477204
\end{verbatim}
\end{kframe}
\end{knitrout}
\item Try writing your own code to calculate the standard error of the
  mean age. Hint: This only requires one line of code. Use online resources
if you cannot remember how the SEM is calculated.
\begin{knitrout}
\definecolor{shadecolor}{rgb}{0.969, 0.969, 0.969}\color{fgcolor}\begin{kframe}
\begin{alltt}
\hlkwd{with}\hlstd{(sports.df,} \hlkwd{sd}\hlstd{(age,}\hlkwc{na.rm} \hlstd{=} \hlnum{TRUE}\hlstd{)}\hlopt{/}\hlkwd{sqrt}\hlstd{(}\hlkwd{length}\hlstd{(age)))}
\end{alltt}
\begin{verbatim}
[1] 0.5477204
\end{verbatim}
\end{kframe}
\end{knitrout}

\end{enumerate}
 
\section{Subsetting datasets}
\label{sec:sub}
\begin{enumerate}[(i)]
\item \label{itm:1way} Produce a one-way frequency table for variable \texttt{q1a}.
\begin{knitrout}
\definecolor{shadecolor}{rgb}{0.969, 0.969, 0.969}\color{fgcolor}\begin{kframe}
\begin{alltt}
\hlkwd{table}\hlstd{(sports.df}\hlopt{$}\hlstd{q1a)}
\end{alltt}
\begin{verbatim}

                      Can´t choose                              Daily 
                               235                                  2 
             Several times a month               Several times a week 
                                66                                  8 
Several times a year or less often 
                               649 
\end{verbatim}
\end{kframe}
\end{knitrout}
\item \label{itm:na} Which participants chose ``Can't
  choose'' for this question?
\begin{knitrout}
\definecolor{shadecolor}{rgb}{0.969, 0.969, 0.969}\color{fgcolor}\begin{kframe}
\begin{alltt}
\hlkwd{which}\hlstd{(sports.df}\hlopt{$}\hlstd{q1a} \hlopt{==} \hlstr{"Can?t choose"}\hlstd{)}
\end{alltt}
\begin{verbatim}
integer(0)
\end{verbatim}
\end{kframe}
\end{knitrout}
\item Now reproduce the frequency table in
  \ref{sec:sub}(\ref{itm:1way}), excluding the participants you identified in
  \ref{sec:sub}(\ref{itm:na}).
\begin{knitrout}
\definecolor{shadecolor}{rgb}{0.969, 0.969, 0.969}\color{fgcolor}\begin{kframe}
\begin{alltt}
\hlstd{excluded} \hlkwb{<-} \hlkwd{which}\hlstd{(sports.df}\hlopt{$}\hlstd{q1a} \hlopt{==} \hlstr{"Can?t choose"}\hlstd{)}
\hlkwd{with}\hlstd{(sports.df[}\hlopt{-}\hlstd{excluded, ],} \hlkwd{table}\hlstd{(q1a))}
\end{alltt}
\begin{verbatim}
q1a
                      Can´t choose                              Daily 
                                 0                                  0 
             Several times a month               Several times a week 
                                 0                                  0 
Several times a year or less often 
                                 0 
\end{verbatim}
\end{kframe}
\end{knitrout}
\item Calculate the mean age of male participants.
\begin{knitrout}
\definecolor{shadecolor}{rgb}{0.969, 0.969, 0.969}\color{fgcolor}\begin{kframe}
\begin{alltt}
\hlkwd{with}\hlstd{(sports.df,} \hlkwd{mean}\hlstd{(age[gender} \hlopt{==} \hlstr{"Male"}\hlstd{],} \hlkwc{na.rm} \hlstd{=} \hlnum{TRUE}\hlstd{))}
\end{alltt}
\begin{verbatim}
[1] 52.88503
\end{verbatim}
\end{kframe}
\end{knitrout}
\item Calculate the mean age of male participants who earn more than
  \$100000 a year.
\begin{knitrout}
\definecolor{shadecolor}{rgb}{0.969, 0.969, 0.969}\color{fgcolor}\begin{kframe}
\begin{alltt}
\hlkwd{with}\hlstd{(sports.df,} \hlkwd{mean}\hlstd{(age[gender} \hlopt{==} \hlstr{"Male"} \hlopt{&}
                         \hlstd{income} \hlopt{==} \hlstr{"> 100 000$"}\hlstd{],} \hlkwc{na.rm} \hlstd{=} \hlnum{TRUE}\hlstd{))}
\end{alltt}
\begin{verbatim}
[1] 51
\end{verbatim}
\end{kframe}
\end{knitrout}
\item Calculate the mean age of European male participants who earn more than
  \$100000 a year.
\begin{knitrout}
\definecolor{shadecolor}{rgb}{0.969, 0.969, 0.969}\color{fgcolor}\begin{kframe}
\begin{alltt}
\hlkwd{with}\hlstd{(sports.df,} \hlkwd{mean}\hlstd{(age[gender} \hlopt{==} \hlstr{"Male"} \hlopt{&}  \hlstd{income} \hlopt{==} \hlstr{"> 100 000$"} \hlopt{&}
                         \hlstd{ethnicity} \hlopt{==} \hlstr{"Europe,White/European"}\hlstd{],} \hlkwc{na.rm} \hlstd{=} \hlnum{TRUE}\hlstd{))}
\end{alltt}
\begin{verbatim}
[1] 52.14815
\end{verbatim}
\end{kframe}
\end{knitrout}
\end{enumerate}
\section{Challenge}
\label{sec:cha}

Modify the function given in {\bf\ref{sec:func}}, so that the function will
return a 95\% confidence interval (with 2 decimal places). Hint: A 95\% confidence interval of a variable \texttt{x} is given by the mean of \texttt{x} $\pm$ 1.96 $\times$ SEM of \texttt{x}. You might find the \texttt{paste()}
function useful.

\begin{knitrout}
\definecolor{shadecolor}{rgb}{0.969, 0.969, 0.969}\color{fgcolor}\begin{kframe}
\begin{alltt}
\hlstd{mystder} \hlkwb{<-} \hlkwa{function}\hlstd{(}\hlkwc{x}\hlstd{)\{}
       \hlstd{mymean} \hlkwb{<-} \hlkwd{mean}\hlstd{(x,} \hlkwc{na.rm} \hlstd{=} \hlnum{TRUE}\hlstd{)}
       \hlstd{mysd} \hlkwb{<-} \hlkwd{sd}\hlstd{(x,} \hlkwc{na.rm} \hlstd{=} \hlnum{TRUE}\hlstd{)}
       \hlstd{n} \hlkwb{<-} \hlkwd{length}\hlstd{(x)}
       \hlstd{mystder} \hlkwb{=} \hlstd{mysd}\hlopt{/}\hlkwd{sqrt}\hlstd{(n)}
       \hlstd{upperCI} \hlkwb{=} \hlkwd{round}\hlstd{(mymean} \hlopt{+} \hlnum{1.96}\hlopt{*}\hlstd{mystder,} \hlnum{2}\hlstd{)}
       \hlstd{lowerCI} \hlkwb{=} \hlkwd{round}\hlstd{(mymean} \hlopt{-} \hlnum{1.96}\hlopt{*}\hlstd{mystder,} \hlnum{2}\hlstd{)}
       \hlkwd{paste}\hlstd{(}\hlstr{"("}\hlstd{, lowerCI,} \hlstr{" , "}\hlstd{, upperCI,} \hlstr{")"}\hlstd{,} \hlkwc{sep} \hlstd{=} \hlstr{""}\hlstd{)}
\hlstd{\}}
\hlkwd{mystder}\hlstd{(sports.df}\hlopt{$}\hlstd{age)}
\end{alltt}
\begin{verbatim}
[1] "(50.75 , 52.9)"
\end{verbatim}
\end{kframe}
\end{knitrout}

\end{document}
