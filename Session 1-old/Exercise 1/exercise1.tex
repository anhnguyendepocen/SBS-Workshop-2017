

\documentclass[12pt,a4paper]{article}
\usepackage{amsmath}
\usepackage{enumerate}
\usepackage[cm]{fullpage}

\begin{document}
\setlength\parindent{0cm}
\title{\Large{\textbf{Introduction to \texttt{R}}}\\
\textit{Session 1 exercises}}
\author{Statistical Consulting Centre}
\date{1 March, 2017}
\maketitle

\section{Using \textbf{R} as a calculator}
\label{sec:use-r-as}

\begin{enumerate}[(i)]

\item \label{itm:calculator} Find the values of:
  \begin{enumerate}
  \item $1+4$
  \item $2^3 + \frac{4}{\sqrt{34}}$
  \item $\log{30}$
  \item $\log_{10}30$
  \item $|-2|$ \hspace{0.2cm}(Hint: $|x|$ denotes the \emph{absolute
      value} of $x$. Search on Google if you're unsure.)
  \end{enumerate}
\item Now open Rstudio, open a \texttt{R} script clicking \texttt{File}
  $\rightarrow$ \texttt{New} $\rightarrow$ \texttt{R} script.
\item Save this script by clicking \texttt{File} $\rightarrow$
  \texttt{Save As...}.
\item Select a directory/location and save the script. Note: the saved
  script should have \texttt{.r} as extension. For example, if you
  call your file \texttt{exercise one}, then you should save it as
  \texttt{exercise one.r}
\item Copy and paste the code you typed (\emph{not the output, not the
  $>$ symbol, just the code you typed}) at the console for {\bf
    \ref{sec:use-r-as}}(\ref{itm:calculator}) into the \texttt{R}
  script opened in Rstudio.
\item Submit your entire script at once to the \texttt{R}
  Console by highlighting all codes and pressing Ctrl $+$ R.
\item From now on, type all of your code in your \texttt{R} script and
  submit it to the \texttt{R} Console using Ctrl $+$ R.

\end{enumerate}

\section{Reading data into \texttt{R}}
\label{sec:read}

\begin{enumerate}[(i)]

\item We are going to use Leisure Time and Sports Questionnaire done
  by ISSP at 2007 for our exercises. Again we only take a small
  proportion of the survey (\texttt{sports.csv}). Please see the
  questionnaire provided. 
\item Read the data into \texttt{R}, saving it in an object named
  \texttt{sports.df}. 
\item Use \texttt{dim()} and \texttt{head()} to look at some of the
  properties of the dataset you have just read into
  \texttt{R}. \emph{Always} perform these two important checks of your
  data to ensure what you have read into \texttt{R} is as it should
  be. 
\item Calculate the mean and standard deviation of age.
\item Check the frequency of gender.
\item \label{itm:2freq} Produce a two-way frequency table between
  ethnicity and age. 
\item Turn the frequency table in \ref{sec:read}(\ref{itm:2freq})
into column proportions, keep only 1 decimal place.
\item Now turn the frequency table in \ref{sec:read}(\ref{itm:2freq})
  into row proportions, keep only 1 decimal place.
\end{enumerate}
\end{document}
