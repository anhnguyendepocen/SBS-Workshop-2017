

\documentclass[12pt,a4paper]{article}
\usepackage{amsmath}
\usepackage{enumerate}
\usepackage[cm]{fullpage}
\usepackage{graphicx}
\usepackage{Sweave}
\begin{document}
\Sconcordance{concordance:exercise6.tex:exercise6.Rnw:%
1 7 1 1 0 11 1 1 42 3 1 1 19 10 1 1 25 5 1 1 3 5 0 1 2 3 1 1 33 6 1}

\setlength\parindent{0cm}
%\setlength{\oddsidemargin}{0.25cm}
%\setlength{\evensidemargin}{0.25cm}
\title{\Large{\textbf{Introduction to \texttt{R}}}\\
\textit{Session 6 exercises}}
\author{Statistical Consulting Centre}
\date{2 March, 2017}
\maketitle
 
 
\begin{enumerate}
\item Plot the mean nerdy score for each gender (with $\pm$1.96$\times$SE bars), as shown in Figure \ref{fig:stder1}. 
\begin{figure}[h]   
 \centering
\caption{First plot with standard error bars}
  \label{fig:stder1}
\end{figure}
\newpage
\item Now reproduce Figure \ref{fig:stder2}. The graph should have:
\begin{itemize}
\item for Males: solid circles (representing the mean) and $\pm$1.96$\times$SE bars (representing the lower and upper 95\% confidence limits).
\item for Females: solid circles (representing the mean) and $\pm$1.96$\times$SE bars (representing the lower and upper 95\% confidence limits).
\end{itemize}
\begin{figure}[h]   
 \centering
\caption{Second plot with standard error bars}
  \label{fig:stder2}
\end{figure}

\newpage
\item Now read-in the file called \verb|NZmap.csv| in the Data folder and reproduce Figure \ref{fig:stder3}. Hint: create the \texttt{ggplot} object using the following
\begin{Schunk}
\begin{Sinput}
> ggplot(data = nz1.df, aes(x = long, y = lat, group = group, 
+                            fill = Outcome ))
\end{Sinput}
\end{Schunk}
Then, use \verb|geom_polygon()| function.  

\begin{figure}[h]   
 \centering

\caption{Third plot with mapping}
  \label{fig:stder3}
\end{figure}

\end{enumerate}
\end{document}
