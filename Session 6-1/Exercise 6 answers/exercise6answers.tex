

\documentclass[12pt,a4paper]{article}
\usepackage{amsmath}
\usepackage{enumerate}
\usepackage[cm]{fullpage}
\usepackage{graphicx}
\usepackage{Sweave}
\begin{document}
\Sconcordance{concordance:exercise6answers.tex:exercise6answers.Rnw:%
1 7 1 1 0 10 1 1 42 3 1 1 2 1 0 3 1 1 7 5 0 1 5 6 0 1 2 10 1 1 2 1 0 5 %
1 1 7 6 0 1 1 1 7 9 0 1 2 6 1 1 2 1 0 1 2 15 1 1 12 13 0 1 2 6 1}

\setlength\parindent{0cm}
%\setlength{\oddsidemargin}{0.25cm}
%\setlength{\evensidemargin}{0.25cm}
\title{\Large{\textbf{Introduction to \texttt{R}}}\\
\textit{Answers to Session 6 exercises}}
\author{Statistical Consulting Centre}
\date{2 March, 2017}
\maketitle
 
\begin{enumerate}
\item Plot the mean nerdy score for each gender (with $\pm$1.96$\times$SE bars), as shown in Figure \ref{fig:stder1}. 
\begin{figure}[h]   
 \centering
\begin{Schunk}
\begin{Sinput}
> g.m <- with(sports.df, tapply(nerdy.sc, gender, mean, na.rm = TRUE))
> g.sd <- with(sports.df, tapply(nerdy.sc, gender, sd, na.rm = TRUE))
> g.n <- with(sports.df, tapply(nerdy.sc, gender, function(x)sum(!is.na(x))))
> g.stder <- g.sd/sqrt(g.n)
> G.df <- data.frame(
+   Gender = names(g.m),
+   Mean = g.m,
+   Upper = g.m + 1.96 * g.stder,
+   Lower = g.m - 1.96 * g.stder
+   )
> ggplot(G.df, aes(x = Gender, y = Mean)) + 
+   geom_point() +
+    geom_errorbar(aes(ymax = Upper, ymin = Lower),
+                 width = 0.1)
\end{Sinput}
\end{Schunk}
\caption{First plot with standard error bars}
  \label{fig:stder1}
\end{figure}
\newpage
\item Now reproduce Figure \ref{fig:stder2}. The graph should have:
\begin{itemize}
\item for Males: solid circles (representing the mean) and $\pm$1.96$\times$SE bars (representing the lower and upper 95\% confidence limits).
\item for Females: solid circles (representing the mean) and $\pm$1.96$\times$SE bars (representing the lower and upper 95\% confidence limits).
\end{itemize}
\begin{figure}[h]   
 \centering
\begin{Schunk}
\begin{Sinput}
> ga.m <- with(sports.df, tapply(nerdy.sc, list(gender, age.group), mean, na.rm = TRUE))
> ga.sd <- with(sports.df, tapply(nerdy.sc, list(gender, age.group), sd, na.rm = TRUE))
> ga.n <- with(sports.df, tapply(nerdy.sc, list(gender, age.group), function(x)sum(!is.na(x))))
> ga.stder <- ga.sd/sqrt(ga.n)
> ga.upper <-ga.m + 1.96*ga.stder
> ga.lower <-ga.m - 1.96*ga.stder
> GA.df = data.frame(
+   Age.group = factor(rep(colnames(ga.m), 2), levels = colnames(ga.m)),
+   Gender = rep(rownames(ga.m), c(3, 3)),
+   Mean = c(ga.m[1, ], ga.m[2, ]),
+   Upper = c(ga.upper[1, ], ga.upper[2, ]),
+   Lower = c(ga.lower[1, ], ga.lower[2, ])
+ )
> dodge <- position_dodge(width=0.2)
> ggplot(GA.df, aes(x = Age.group, y = Mean,
+                   color = Gender)) +
+   xlab("Age Group")+
+   ylab("Mean total Nerdy scores") +
+   geom_point(position = dodge) +
+   geom_errorbar(aes(ymax = Upper, ymin = Lower),
+                 width = 0.1, position = dodge)
\end{Sinput}
\end{Schunk}
\caption{Second plot with standard error bars}
  \label{fig:stder2}
\end{figure}

\item Now read-in the file NZmap.csv is the Data folder and reproduce Figure \ref{fig:stder3}. 
\begin{figure}[h]   
 \centering
\begin{Schunk}
\begin{Sinput}
> nz1.df <- read.csv("../../Data/NZmap.csv")
> nz1.df$Outcome[nz1.df$NAME_1 == "Auckland"] <- 10
> nz1.df$Outcome[nz1.df$NAME_1 == "Northland"] <- 11
> nz1.df$Outcome[nz1.df$NAME_1 == "Otago"] <- 5
> nz1.df$Outcome[nz1.df$NAME_1 == "Southland"] <- 2
> nz1.df$Outcome[nz1.df$NAME_1 == "Taranaki"] <- 3
> nz1.df$Outcome[nz1.df$NAME_1 == "Waikato"] <- 7
> nz1.df$Outcome[nz1.df$NAME_1 == "Wellington"] <- 8
> nz1.df$Outcome[nz1.df$NAME_1 == "West Coast"] <- 1
> nz1.df$Outcome[nz1.df$NAME_1 == "Bay of Plenty"] <- 9
> nz1.df$Outcome[nz1.df$NAME_1 == "Canterbury"] <- 8
> nz1.df$Outcome[nz1.df$NAME_1 == "Chatham Islands"] <- 0
> nz1.df$Outcome[nz1.df$NAME_1 == "Gisborne"] <- 6
> nz1.df$Outcome[nz1.df$NAME_1 == "Hawke's Bay"] <- 6
> nz1.df$Outcome[nz1.df$NAME_1 == "Manawatu-Wanganui"] <- 1
> nz1.df$Outcome[nz1.df$NAME_1 == "Marlborough"] <- 25
> nz1.df$Outcome[nz1.df$NAME_1 == "Nelson"] <- 1
> ggplot(data = nz1.df, aes(x = long, y = lat, group = group, 
+                            fill = Outcome )) +
+   geom_polygon(show.legend = TRUE) +
+   theme_bw() +
+   ylim(4640747, 6255441) +
+   xlim( -321280.2, 640000)+
+   theme(
+     axis.title = element_blank(),
+     axis.text = element_blank(),
+     axis.ticks = element_blank()
+   )
\end{Sinput}
\end{Schunk}

\caption{Third plot with mapping}
  \label{fig:stder3}
\end{figure}

\end{enumerate}
\end{document}
