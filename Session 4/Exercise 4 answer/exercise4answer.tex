

\documentclass[12pt,a4paper]{article}\usepackage[]{graphicx}\usepackage[]{color}
%% maxwidth is the original width if it is less than linewidth
%% otherwise use linewidth (to make sure the graphics do not exceed the margin)
\makeatletter
\def\maxwidth{ %
  \ifdim\Gin@nat@width>\linewidth
    \linewidth
  \else
    \Gin@nat@width
  \fi
}
\makeatother

\definecolor{fgcolor}{rgb}{0.345, 0.345, 0.345}
\newcommand{\hlnum}[1]{\textcolor[rgb]{0.686,0.059,0.569}{#1}}%
\newcommand{\hlstr}[1]{\textcolor[rgb]{0.192,0.494,0.8}{#1}}%
\newcommand{\hlcom}[1]{\textcolor[rgb]{0.678,0.584,0.686}{\textit{#1}}}%
\newcommand{\hlopt}[1]{\textcolor[rgb]{0,0,0}{#1}}%
\newcommand{\hlstd}[1]{\textcolor[rgb]{0.345,0.345,0.345}{#1}}%
\newcommand{\hlkwa}[1]{\textcolor[rgb]{0.161,0.373,0.58}{\textbf{#1}}}%
\newcommand{\hlkwb}[1]{\textcolor[rgb]{0.69,0.353,0.396}{#1}}%
\newcommand{\hlkwc}[1]{\textcolor[rgb]{0.333,0.667,0.333}{#1}}%
\newcommand{\hlkwd}[1]{\textcolor[rgb]{0.737,0.353,0.396}{\textbf{#1}}}%
\let\hlipl\hlkwb

\usepackage{framed}
\makeatletter
\newenvironment{kframe}{%
 \def\at@end@of@kframe{}%
 \ifinner\ifhmode%
  \def\at@end@of@kframe{\end{minipage}}%
  \begin{minipage}{\columnwidth}%
 \fi\fi%
 \def\FrameCommand##1{\hskip\@totalleftmargin \hskip-\fboxsep
 \colorbox{shadecolor}{##1}\hskip-\fboxsep
     % There is no \\@totalrightmargin, so:
     \hskip-\linewidth \hskip-\@totalleftmargin \hskip\columnwidth}%
 \MakeFramed {\advance\hsize-\width
   \@totalleftmargin\z@ \linewidth\hsize
   \@setminipage}}%
 {\par\unskip\endMakeFramed%
 \at@end@of@kframe}
\makeatother

\definecolor{shadecolor}{rgb}{.97, .97, .97}
\definecolor{messagecolor}{rgb}{0, 0, 0}
\definecolor{warningcolor}{rgb}{1, 0, 1}
\definecolor{errorcolor}{rgb}{1, 0, 0}
\newenvironment{knitrout}{}{} % an empty environment to be redefined in TeX

\usepackage{alltt}
\usepackage{amsmath}
\usepackage{enumerate}
\usepackage[cm]{fullpage}
\IfFileExists{upquote.sty}{\usepackage{upquote}}{}
\begin{document}
\setlength\parindent{0cm}
%\setlength{\oddsidemargin}{0.25cm}
%\setlength{\evensidemargin}{0.25cm}
\title{\Large{\textbf{Introduction to \texttt{R}}}\\
\textit{Answers to Session 4 exercises}}
\author{Statistical Consulting Centre}
\date{1 March, 2017}
\maketitle
 
 

\begin{enumerate}
\item \label{itm:q1a} Generate a one-way frequency table for \texttt{q1a}.
\begin{knitrout}
\definecolor{shadecolor}{rgb}{0.969, 0.969, 0.969}\color{fgcolor}\begin{kframe}
\begin{alltt}
\hlkwd{table}\hlstd{(sports.df}\hlopt{$}\hlstd{q1a)}
\end{alltt}
\begin{verbatim}

                             Daily               Several times a week 
                                 2                                  8 
             Several times a month Several times a year or less often 
                                66                                649 
\end{verbatim}
\end{kframe}
\end{knitrout}
\item Create a new variable called \texttt{q1a.sc} (meaning \emph{q1a score}), where \texttt{q1a.sc} is of type numeric/integer rather than of type \texttt{factor}.
\begin{knitrout}
\definecolor{shadecolor}{rgb}{0.969, 0.969, 0.969}\color{fgcolor}\begin{kframe}
\begin{alltt}
\hlstd{q1a.sc} \hlkwb{<-} \hlkwd{as.numeric}\hlstd{(sports.df}\hlopt{$}\hlstd{q1a)}
\end{alltt}
\end{kframe}
\end{knitrout}
\item \label{itm:last} Generate a one-way frequency table of \texttt{q1a.sc} and compare it with the one you generated in question \ref{itm:q1a}. Their frequencies should be identical.
\begin{knitrout}
\definecolor{shadecolor}{rgb}{0.969, 0.969, 0.969}\color{fgcolor}\begin{kframe}
\begin{alltt}
\hlkwd{table}\hlstd{(q1a.sc)}
\end{alltt}
\begin{verbatim}
q1a.sc
  1   2   3   4 
  2   8  66 649 
\end{verbatim}
\end{kframe}
\end{knitrout}
\item Repeat the steps in questions \ref{itm:q1a} -- \ref{itm:last} for variables \texttt{q1b} to \texttt{q1e}, thereby creating new variables \texttt{q1b.sc} -- \texttt{q1e}.
\begin{knitrout}
\definecolor{shadecolor}{rgb}{0.969, 0.969, 0.969}\color{fgcolor}\begin{kframe}
\begin{alltt}
\hlkwd{table}\hlstd{(sports.df}\hlopt{$}\hlstd{q1b)}
\end{alltt}
\begin{verbatim}

                             Daily               Several times a week 
                                 3                                  4 
             Several times a month Several times a year or less often 
                                63                                649 
\end{verbatim}
\begin{alltt}
\hlstd{q1b.sc} \hlkwb{<-} \hlkwd{as.numeric}\hlstd{(sports.df}\hlopt{$}\hlstd{q1b)}
\hlkwd{table}\hlstd{(q1b.sc)}
\end{alltt}
\begin{verbatim}
q1b.sc
  1   2   3   4 
  3   4  63 649 
\end{verbatim}
\begin{alltt}
\hlkwd{table}\hlstd{(sports.df}\hlopt{$}\hlstd{q1c)}
\end{alltt}
\begin{verbatim}

                             Daily               Several times a week 
                                35                                277 
             Several times a month Several times a year or less often 
                               437                                194 
\end{verbatim}
\begin{alltt}
\hlstd{q1c.sc} \hlkwb{<-} \hlkwd{as.numeric}\hlstd{(sports.df}\hlopt{$}\hlstd{q1c)}
\hlkwd{table}\hlstd{(q1c.sc)}
\end{alltt}
\begin{verbatim}
q1c.sc
  1   2   3   4 
 35 277 437 194 
\end{verbatim}
\begin{alltt}
\hlkwd{table}\hlstd{(sports.df}\hlopt{$}\hlstd{q1d)}
\end{alltt}
\begin{verbatim}

                             Daily               Several times a week 
                                62                                281 
             Several times a month Several times a year or less often 
                               440                                177 
\end{verbatim}
\begin{alltt}
\hlstd{q1d.sc} \hlkwb{<-} \hlkwd{as.numeric}\hlstd{(sports.df}\hlopt{$}\hlstd{q1d)}
\hlkwd{table}\hlstd{(q1d.sc)}
\end{alltt}
\begin{verbatim}
q1d.sc
  1   2   3   4 
 62 281 440 177 
\end{verbatim}
\begin{alltt}
\hlkwd{table}\hlstd{(sports.df}\hlopt{$}\hlstd{q1e)}
\end{alltt}
\begin{verbatim}

                             Daily               Several times a week 
                               244                                371 
             Several times a month Several times a year or less often 
                               187                                110 
\end{verbatim}
\begin{alltt}
\hlstd{q1e.sc} \hlkwb{<-} \hlkwd{as.numeric}\hlstd{(sports.df}\hlopt{$}\hlstd{q1e)}
\hlkwd{table}\hlstd{(q1e.sc)}
\end{alltt}
\begin{verbatim}
q1e.sc
  1   2   3   4 
244 371 187 110 
\end{verbatim}
\end{kframe}
\end{knitrout}
\item Create a data frame called \texttt{mean.df} containing all five score variables (\texttt{q1a.sc} -- \texttt{q1e.sc}) which you've created. 
\begin{knitrout}
\definecolor{shadecolor}{rgb}{0.969, 0.969, 0.969}\color{fgcolor}\begin{kframe}
\begin{alltt}
\hlstd{mean.df} \hlkwb{<-} \hlkwd{data.frame}\hlstd{(}\hlkwd{cbind}\hlstd{(q1a.sc, q1b.sc, q1c.sc, q1d.sc, q1e.sc))}
\hlcom{## Always check whether mean.df contains what it supposed to contain}
\hlkwd{dim}\hlstd{(mean.df)}
\end{alltt}
\begin{verbatim}
[1] 996   5
\end{verbatim}
\begin{alltt}
\hlkwd{head}\hlstd{(mean.df)}
\end{alltt}
\begin{verbatim}
  q1a.sc q1b.sc q1c.sc q1d.sc q1e.sc
1      4      4      4      2      1
2      4     NA      3      4      4
3     NA     NA     NA      3      1
4      4      4      2      3     NA
5      4      4      2      3      2
6     NA      4      2      1      1
\end{verbatim}
\end{kframe}
\end{knitrout}
\item Use \texttt{apply()} on \texttt{mean.df} to calculate each participant's mean score across variables \texttt{q1a.sc} -- \texttt{q1e.sc}. Name this new variable \texttt{nerdy.sc}, meaning \emph{nerdy score}.
\begin{knitrout}
\definecolor{shadecolor}{rgb}{0.969, 0.969, 0.969}\color{fgcolor}\begin{kframe}
\begin{alltt}
\hlstd{nerdy.sc} \hlkwb{<-} \hlkwd{apply}\hlstd{(mean.df,} \hlnum{1}\hlstd{, mean,} \hlkwc{na.rm} \hlstd{=} \hlnum{TRUE}\hlstd{)}
\end{alltt}
\end{kframe}
\end{knitrout}
\item Add the variable \texttt{nerdy.sc} to the \texttt{mean.df} data frame and use \texttt{summary()} to generate the five-number-summary of all \emph{six} variables in \texttt{mean.df}.
\begin{knitrout}
\definecolor{shadecolor}{rgb}{0.969, 0.969, 0.969}\color{fgcolor}\begin{kframe}
\begin{alltt}
\hlstd{mean.df}\hlopt{$}\hlstd{nerdy.sc} \hlkwb{<-} \hlstd{nerdy.sc}
\hlkwd{head}\hlstd{(mean.df)}
\end{alltt}
\begin{verbatim}
  q1a.sc q1b.sc q1c.sc q1d.sc q1e.sc nerdy.sc
1      4      4      4      2      1     3.00
2      4     NA      3      4      4     3.75
3     NA     NA     NA      3      1     2.00
4      4      4      2      3     NA     3.25
5      4      4      2      3      2     3.00
6     NA      4      2      1      1     2.00
\end{verbatim}
\begin{alltt}
\hlkwd{apply}\hlstd{(mean.df,} \hlnum{2}\hlstd{, summary)}
\end{alltt}
\begin{verbatim}
         q1a.sc  q1b.sc q1c.sc q1d.sc q1e.sc nerdy.sc
Min.      1.000   1.000  1.000  1.000  1.000    1.000
1st Qu.   4.000   4.000  2.000  2.000  1.000    2.800
Median    4.000   4.000  3.000  3.000  2.000    3.000
Mean      3.879   3.889  2.838  2.762  2.179    3.006
3rd Qu.   4.000   4.000  3.000  3.000  3.000    3.333
Max.      4.000   4.000  4.000  4.000  4.000    4.000
NA's    271.000 277.000 53.000 36.000 84.000    7.000
\end{verbatim}
\end{kframe}
\end{knitrout}
\item Add the columns of \texttt{nerdy.sc} to \texttt{sports.df} for future use.
\begin{knitrout}
\definecolor{shadecolor}{rgb}{0.969, 0.969, 0.969}\color{fgcolor}\begin{kframe}
\begin{alltt}
\hlstd{sports.df}\hlopt{$}\hlstd{nerdy.sc} \hlkwb{<-} \hlstd{nerdy.sc}
\end{alltt}
\end{kframe}
\end{knitrout}
\item \label{itm:income} Use \texttt{tapply()} to calculate the mean nerdy score for all ten income levels.
\begin{knitrout}
\definecolor{shadecolor}{rgb}{0.969, 0.969, 0.969}\color{fgcolor}\begin{kframe}
\begin{alltt}
\hlkwd{with}\hlstd{(sports.df,} \hlkwd{tapply}\hlstd{(nerdy.sc, income, mean,} \hlkwc{na.rm} \hlstd{=} \hlnum{TRUE}\hlstd{))}
\end{alltt}
\begin{verbatim}
      > 100 000$  10 000$-15 000$  15 000$-20 000$  20 000$-25 000$ 
        2.951323         2.884722         3.092177         2.930044 
 25 000$-30 000$  30 000$-40 000$  40 000$-50 000$           5 000$ 
        3.075231         3.000517         3.070673         3.026471 
 50 000$-70 000$ 70 000$-100 000$ 
        2.983465         3.063077 
\end{verbatim}
\end{kframe}
\end{knitrout}
\item \label{itm:level} Income level 1 is shown first in the output of question \ref{itm:income} while income level 10 is shown last. Do you agree with \texttt{R}'s default ordering of \texttt{income} levels? If not, appropriately order the levels of \texttt{Income}.
\begin{knitrout}
\definecolor{shadecolor}{rgb}{0.969, 0.969, 0.969}\color{fgcolor}\begin{kframe}
\begin{alltt}
\hlstd{sports.df}\hlopt{$}\hlstd{income} \hlkwb{=} \hlkwd{factor}\hlstd{(sports.df}\hlopt{$}\hlstd{income,} \hlkwc{levels} \hlstd{=} \hlkwd{c}\hlstd{(}\hlstr{"5 000$"}\hlstd{,} \hlstr{"10 000$-15 000$"}\hlstd{,} \hlstr{"15 000$-20 000$"}\hlstd{,}
                                             \hlstr{"20 000$-25 000$"}\hlstd{,} \hlstr{"25 000$-30 000$"}\hlstd{,} \hlstr{"30 000$-40 000$"}\hlstd{,}
                                             \hlstr{"40 000$-50 000$"}\hlstd{,} \hlstr{"50 000$-70 000$"}\hlstd{,} \hlstr{"70 000$-100 000$"}\hlstd{,}
                                             \hlstr{"> 100 000$"}\hlstd{))}
\end{alltt}
\end{kframe}
\end{knitrout}
\item Repeat question \ref{itm:income} to check that {\em your chosen} ordering of \texttt{Income} levels has been correctly set.
\begin{knitrout}
\definecolor{shadecolor}{rgb}{0.969, 0.969, 0.969}\color{fgcolor}\begin{kframe}
\begin{alltt}
\hlkwd{with}\hlstd{(sports.df,} \hlkwd{tapply}\hlstd{(nerdy.sc, income, mean,} \hlkwc{na.rm} \hlstd{=} \hlnum{TRUE}\hlstd{))}
\end{alltt}
\begin{verbatim}
          5 000$  10 000$-15 000$  15 000$-20 000$  20 000$-25 000$ 
        3.026471         2.884722         3.092177         2.930044 
 25 000$-30 000$  30 000$-40 000$  40 000$-50 000$  50 000$-70 000$ 
        3.075231         3.000517         3.070673         2.983465 
70 000$-100 000$       > 100 000$ 
        3.063077         2.951323 
\end{verbatim}
\end{kframe}
\end{knitrout}
\item You were introduced to the following function, \texttt{mytab()}, in the Session 4 lecture slides.
\begin{knitrout}
\definecolor{shadecolor}{rgb}{0.969, 0.969, 0.969}\color{fgcolor}\begin{kframe}
\begin{alltt}
\hlstd{mytab} \hlkwb{<-} \hlkwa{function}\hlstd{(}\hlkwc{someinput}\hlstd{)\{}
 \hlstd{n} \hlkwb{<-} \hlkwd{length}\hlstd{(someinput)}
 \hlstd{n.missing} \hlkwb{<-} \hlkwd{na.check}\hlstd{(someinput)}
 \hlstd{n.complete} \hlkwb{<-} \hlstd{n} \hlopt{-} \hlstd{n.missing}
 \hlstd{mymean} \hlkwb{<-} \hlkwd{round}\hlstd{(}\hlkwd{mean}\hlstd{(someinput,} \hlkwc{na.rm} \hlstd{=} \hlnum{TRUE}\hlstd{),} \hlnum{2}\hlstd{)}
 \hlstd{mysd} \hlkwb{<-} \hlkwd{round}\hlstd{(}\hlkwd{sd}\hlstd{(someinput,} \hlkwc{na.rm} \hlstd{=} \hlnum{TRUE}\hlstd{),} \hlnum{2}\hlstd{)}
 \hlstd{mystder} \hlkwb{<-} \hlkwd{round}\hlstd{(mysd}\hlopt{/}\hlkwd{sqrt}\hlstd{(n.complete),} \hlnum{2}\hlstd{)}
 \hlstd{Lower.CI} \hlkwb{<-} \hlkwd{round}\hlstd{(mymean} \hlopt{-} \hlnum{1.96}\hlopt{*}\hlstd{mystder,} \hlnum{2}\hlstd{)}
 \hlstd{Upper.CI} \hlkwb{<-} \hlkwd{round}\hlstd{(mymean} \hlopt{+} \hlnum{1.96}\hlopt{*}\hlstd{mystder,} \hlnum{2}\hlstd{)}
 \hlkwd{c}\hlstd{(}\hlkwc{Complete.obs} \hlstd{= n.complete,} \hlkwc{Missing.obs} \hlstd{= n.missing,}
   \hlkwc{Mean} \hlstd{= mymean,} \hlkwc{Std.Error} \hlstd{= mystder,}
   \hlkwc{Lower.CI} \hlstd{= Lower.CI,} \hlkwc{Upper.CI} \hlstd{= Upper.CI)}
\hlstd{\}}
\end{alltt}
\end{kframe}
\end{knitrout}
It depends on the \texttt{na.check()} function, defined earlier, to calculate the number of missing values, i.e., \texttt{mytab()} depends on the availability of \texttt{na.check()} in order for it to work. Modify \texttt{mytab()} so it does {\em no longer} depends on \texttt{na.check()} to calculate the number of missing values. Let's call the modified function \texttt{mytab1()}.
\begin{knitrout}
\definecolor{shadecolor}{rgb}{0.969, 0.969, 0.969}\color{fgcolor}\begin{kframe}
\begin{alltt}
\hlstd{mytab1} \hlkwb{<-} \hlkwa{function}\hlstd{(}\hlkwc{someinput}\hlstd{)\{}
 \hlstd{n} \hlkwb{<-} \hlkwd{length}\hlstd{(someinput)}
 \hlstd{n.missing} \hlkwb{<-} \hlkwd{length}\hlstd{(}\hlkwd{which}\hlstd{(}\hlkwd{is.na}\hlstd{(someinput)))}
 \hlstd{n.complete} \hlkwb{<-} \hlstd{n} \hlopt{-} \hlstd{n.missing}
 \hlstd{mymean} \hlkwb{<-} \hlkwd{round}\hlstd{(}\hlkwd{mean}\hlstd{(someinput,} \hlkwc{na.rm} \hlstd{=} \hlnum{TRUE}\hlstd{),} \hlnum{2}\hlstd{)}
 \hlstd{mysd} \hlkwb{<-} \hlkwd{round}\hlstd{(}\hlkwd{sd}\hlstd{(someinput,} \hlkwc{na.rm} \hlstd{=} \hlnum{TRUE}\hlstd{),} \hlnum{2}\hlstd{)}
 \hlstd{mystder} \hlkwb{<-} \hlkwd{round}\hlstd{(mysd}\hlopt{/}\hlkwd{sqrt}\hlstd{(n.complete),} \hlnum{2}\hlstd{)}
 \hlstd{Lower.CI} \hlkwb{<-} \hlkwd{round}\hlstd{(mymean} \hlopt{-} \hlnum{1.96}\hlopt{*}\hlstd{mystder,} \hlnum{2}\hlstd{)}
 \hlstd{Upper.CI} \hlkwb{<-} \hlkwd{round}\hlstd{(mymean} \hlopt{+} \hlnum{1.96}\hlopt{*}\hlstd{mystder,} \hlnum{2}\hlstd{)}
 \hlkwd{c}\hlstd{(}\hlkwc{Complete.obs} \hlstd{= n.complete,} \hlkwc{Missing.obs} \hlstd{= n.missing,}
   \hlkwc{Mean} \hlstd{= mymean,} \hlkwc{Std.Error} \hlstd{= mystder,}
   \hlkwc{Lower.CI} \hlstd{= Lower.CI,} \hlkwc{Upper.CI} \hlstd{= Upper.CI)}
\hlstd{\}}
\end{alltt}
\end{kframe}
\end{knitrout}
\item Use \texttt{mytab1()} to produce a summary table for all six variables in \texttt{mean.df}.
\begin{knitrout}
\definecolor{shadecolor}{rgb}{0.969, 0.969, 0.969}\color{fgcolor}\begin{kframe}
\begin{alltt}
\hlkwd{apply}\hlstd{(mean.df,} \hlnum{2}\hlstd{, mytab1)}
\end{alltt}
\begin{verbatim}
             q1a.sc q1b.sc q1c.sc q1d.sc q1e.sc nerdy.sc
Complete.obs 725.00 719.00 943.00 960.00 912.00   989.00
Missing.obs  271.00 277.00  53.00  36.00  84.00     7.00
Mean           3.88   3.89   2.84   2.76   2.18     3.01
Std.Error      0.01   0.01   0.03   0.03   0.03     0.02
Lower.CI       3.86   3.87   2.78   2.70   2.12     2.97
Upper.CI       3.90   3.91   2.90   2.82   2.24     3.05
\end{verbatim}
\end{kframe}
\end{knitrout}
\item Use \texttt{mytab1()} to produce a summary table of nerdy scores for all ten income levels.
\begin{knitrout}
\definecolor{shadecolor}{rgb}{0.969, 0.969, 0.969}\color{fgcolor}\begin{kframe}
\begin{alltt}
\hlkwd{tapply}\hlstd{(mean.df}\hlopt{$}\hlstd{nerdy.sc, sports.df}\hlopt{$}\hlstd{income, mytab1)}
\end{alltt}
\begin{verbatim}
$`5 000$`
Complete.obs  Missing.obs         Mean    Std.Error     Lower.CI 
       85.00         1.00         3.03         0.05         2.93 
    Upper.CI 
        3.13 

$`10 000$-15 000$`
Complete.obs  Missing.obs         Mean    Std.Error     Lower.CI 
      120.00         0.00         2.88         0.05         2.78 
    Upper.CI 
        2.98 

$`15 000$-20 000$`
Complete.obs  Missing.obs         Mean    Std.Error     Lower.CI 
       98.00         0.00         3.09         0.05         2.99 
    Upper.CI 
        3.19 

$`20 000$-25 000$`
Complete.obs  Missing.obs         Mean    Std.Error     Lower.CI 
       76.00         0.00         2.93         0.05         2.83 
    Upper.CI 
        3.03 

$`25 000$-30 000$`
Complete.obs  Missing.obs         Mean    Std.Error     Lower.CI 
       72.00         1.00         3.08         0.05         2.98 
    Upper.CI 
        3.18 

$`30 000$-40 000$`
Complete.obs  Missing.obs         Mean    Std.Error     Lower.CI 
      129.00         1.00         3.00         0.05         2.90 
    Upper.CI 
        3.10 

$`40 000$-50 000$`
Complete.obs  Missing.obs         Mean    Std.Error     Lower.CI 
      104.00         2.00         3.07         0.05         2.97 
    Upper.CI 
        3.17 

$`50 000$-70 000$`
Complete.obs  Missing.obs         Mean    Std.Error     Lower.CI 
      127.00         1.00         2.98         0.05         2.88 
    Upper.CI 
        3.08 

$`70 000$-100 000$`
Complete.obs  Missing.obs         Mean    Std.Error     Lower.CI 
       65.00         0.00         3.06         0.06         2.94 
    Upper.CI 
        3.18 

$`> 100 000$`
Complete.obs  Missing.obs         Mean    Std.Error     Lower.CI 
       63.00         0.00         2.95         0.06         2.83 
    Upper.CI 
        3.07 
\end{verbatim}
\end{kframe}
\end{knitrout}
\end{enumerate}
 
\end{document}
