\documentclass{beamer}\usepackage[]{graphicx}\usepackage[]{color}
%% maxwidth is the original width if it is less than linewidth
%% otherwise use linewidth (to make sure the graphics do not exceed the margin)
\makeatletter
\def\maxwidth{ %
  \ifdim\Gin@nat@width>\linewidth
    \linewidth
  \else
    \Gin@nat@width
  \fi
}
\makeatother

\definecolor{fgcolor}{rgb}{0, 0, 0}
\newcommand{\hlnum}[1]{\textcolor[rgb]{0.533,0,0.133}{#1}}%
\newcommand{\hlstr}[1]{\textcolor[rgb]{0.667,0.267,0}{#1}}%
\newcommand{\hlcom}[1]{\textcolor[rgb]{1,0.533,0}{#1}}%
\newcommand{\hlopt}[1]{\textcolor[rgb]{0,0,0}{\textbf{#1}}}%
\newcommand{\hlstd}[1]{\textcolor[rgb]{0,0,0}{#1}}%
\newcommand{\hlkwa}[1]{\textcolor[rgb]{0.4,0.067,0.067}{\textbf{#1}}}%
\newcommand{\hlkwb}[1]{\textcolor[rgb]{0,0,0.4}{\textbf{#1}}}%
\newcommand{\hlkwc}[1]{\textcolor[rgb]{0,0,0.4}{#1}}%
\newcommand{\hlkwd}[1]{\textcolor[rgb]{0,0.267,0.4}{#1}}%
\let\hlipl\hlkwb

\usepackage{framed}
\makeatletter
\newenvironment{kframe}{%
 \def\at@end@of@kframe{}%
 \ifinner\ifhmode%
  \def\at@end@of@kframe{\end{minipage}}%
  \begin{minipage}{\columnwidth}%
 \fi\fi%
 \def\FrameCommand##1{\hskip\@totalleftmargin \hskip-\fboxsep
 \colorbox{shadecolor}{##1}\hskip-\fboxsep
     % There is no \\@totalrightmargin, so:
     \hskip-\linewidth \hskip-\@totalleftmargin \hskip\columnwidth}%
 \MakeFramed {\advance\hsize-\width
   \@totalleftmargin\z@ \linewidth\hsize
   \@setminipage}}%
 {\par\unskip\endMakeFramed%
 \at@end@of@kframe}
\makeatother

\definecolor{shadecolor}{rgb}{.97, .97, .97}
\definecolor{messagecolor}{rgb}{0, 0, 0}
\definecolor{warningcolor}{rgb}{1, 0, 1}
\definecolor{errorcolor}{rgb}{1, 0, 0}
\newenvironment{knitrout}{}{} % an empty environment to be redefined in TeX

\usepackage{alltt}
\newenvironment{changemargin}[2]{%
\begin{list}{}{%
\setlength{\topsep}{0pt}%
\setlength{\leftmargin}{#1}%
\setlength{\rightmargin}{#2}%
\setlength{\listparindent}{\parindent}%
\setlength{\itemindent}{\parindent}%
\setlength{\parsep}{\parskip}%
}%
\item[]}{\end{list}}
\usepackage{graphicx}
\usepackage{amsmath}
\usepackage{url}
\usetheme{Madrid}
\usecolortheme{beaver}
\setbeamertemplate{navigation symbols}{}
\titlegraphic{\includegraphics[width=5cm]{..//..//S-DS-VC-RGB.png}}
%\logo{\includegraphics[width=0.1\textwidth,keepaspectratio]{..//..//UOACrest.png}}
\author[SCC]{Statistical Consulting Centre}%\\
\institute[\href{mailto:consulting@stat.auckland.ac.nz}
  {consulting@stat.auckland.ac.nz}]{\href{mailto:consulting@stat.auckland.ac.nz}
  {consulting@stat.auckland.ac.nz}\\
%Statistical Consulting Centre\\
The Department of Statistics\\
The University of Auckland}
\title[Session 4 -- Data Exploration]{NZSSN Courses: Introduction to \texttt{R}}
\subtitle{Session 4 -- Data Exploration}
\date{1 March, 2017}
\IfFileExists{upquote.sty}{\usepackage{upquote}}{}
\begin{document}
%\SweaveOpts{concordance=TRUE}
\maketitle

\begin{frame}[fragile]
  \frametitle{WHOQOL}

\begin{itemize}
\item Questionnaire usually comprise many items.
\item Sometimes we add the likert scale of a set of questions (from the same section) to obtain a score. Then we will assume it's a measure of a certain aspect.
\item For example, World Health Organisation Quality of Life (WHOQOL) measures. It develops three scores (physical score, psychosocial score and total generic score) from a large number of questions.
\item Suppose we add the likert scale from Q1 to Q4 to get a total score, and let's call it the "Feminist score"
\end{itemize}
\end{frame}

\begin{frame}[fragile]
\frametitle{First step}
Convert responses to Q1 -- Q4 to likert scale.\\
\vspace{2mm}
We can do this using nested \texttt{ifelse()} statements, e.g.
\vspace{-2mm}
\begin{knitrout}
\definecolor{shadecolor}{rgb}{0.965, 0.965, 0.965}\color{fgcolor}\begin{kframe}
\begin{alltt}
\hlstd{Q1.lik}\hlkwb{<-} \hlkwd{with}\hlstd{(issp.df,}
            \hlkwd{ifelse}\hlstd{(Q1} \hlopt{==} \hlstr{"strongly agree"}\hlstd{,} \hlnum{1}\hlstd{,}
            \hlkwd{ifelse}\hlstd{(Q1} \hlopt{==} \hlstr{"agree"}\hlstd{,} \hlnum{2}\hlstd{,}
            \hlkwd{ifelse}\hlstd{(Q1} \hlopt{==} \hlstr{"neither agree nor dis"}\hlstd{,} \hlnum{3}\hlstd{,}
            \hlkwd{ifelse}\hlstd{(Q1} \hlopt{==} \hlstr{"disagree"}\hlstd{,} \hlnum{4}\hlstd{,}
            \hlkwd{ifelse}\hlstd{(Q1} \hlopt{==} \hlstr{"strongly disagree"}\hlstd{,} \hlnum{5}\hlstd{,} \hlnum{NA}\hlstd{))))))}
\end{alltt}
\end{kframe}
\end{knitrout}
\vspace{-2mm}
\begin{knitrout}
\definecolor{shadecolor}{rgb}{0.965, 0.965, 0.965}\color{fgcolor}\begin{kframe}
\begin{alltt}
\hlcom{# Generate one-way frequency table for likert scale}
\hlkwd{table}\hlstd{(Q1.lik)}
\end{alltt}
\end{kframe}
\end{knitrout}
\vspace{-7mm}
\begin{knitrout}
\definecolor{shadecolor}{rgb}{0.965, 0.965, 0.965}\color{fgcolor}\begin{kframe}
\begin{verbatim}
Q1.lik
  1   2   3   4   5 
 98 297 331 263  21 
\end{verbatim}
\end{kframe}
\end{knitrout}
\end{frame}

\begin{frame}[fragile]
\frametitle{Easier way}
Recall that \texttt{Q1} -- \texttt{Q4} are now factors. We can check this using \texttt{R}'s \text{str}ucture function, i.e.
\begin{knitrout}
\definecolor{shadecolor}{rgb}{0.965, 0.965, 0.965}\color{fgcolor}\begin{kframe}
\begin{alltt}
\hlkwd{str}\hlstd{(issp.df[,} \hlkwd{c}\hlstd{(}\hlstr{"Q1"}\hlstd{,} \hlstr{"Q2"}\hlstd{,} \hlstr{"Q3"}\hlstd{,} \hlstr{"Q4"}\hlstd{)])}
\end{alltt}
\begin{verbatim}
'data.frame':	1047 obs. of  4 variables:
 $ Q1: Factor w/ 5 levels "strongly agree",..: 4 5 4 NA 4 4 1 3 NA 4 ...
 $ Q2: Factor w/ 5 levels "strongly agree",..: 2 3 5 4 3 4 3 5 NA 5 ...
 $ Q3: Factor w/ 5 levels "strongly agree",..: 3 4 5 4 4 4 4 5 NA 5 ...
 $ Q4: Factor w/ 5 levels "strongly agree",..: 2 2 2 2 2 2 2 1 NA 2 ...
\end{verbatim}
\end{kframe}
\end{knitrout}
We can look at the levels of \texttt{Q1}, e.g.,
\vspace{-2mm}
\begin{knitrout}
\definecolor{shadecolor}{rgb}{0.965, 0.965, 0.965}\color{fgcolor}\begin{kframe}
\begin{alltt}
\hlkwd{levels}\hlstd{(issp.df}\hlopt{$}\hlstd{Q1)}
\end{alltt}
\end{kframe}
\end{knitrout}
\vspace{-7mm}
\begin{knitrout}
\definecolor{shadecolor}{rgb}{0.965, 0.965, 0.965}\color{fgcolor}\begin{kframe}
\begin{verbatim}
[1] "strongly agree"        "agree"                
[3] "neither agree nor dis" "disagree"             
[5] "strongly disagree"    
\end{verbatim}
\end{kframe}
\end{knitrout}
\end{frame}

\begin{frame}[fragile]
\frametitle{\texttt{as.numeric()}}
Integer values assigned to levels of \texttt{Q1}: \texttt{strongly agree = 1}, \texttt{agree = 2}, ..., \texttt{strongly disagree = 5}. We defined this order using the \texttt{levels} argument in \texttt{factor}.\\
\vspace{2mm}
\texttt{as.numeric()} uses this \texttt{factor} property to convert \texttt{Q1} to type \texttt{numeric}. 
\vspace{-2mm}
\begin{knitrout}
\definecolor{shadecolor}{rgb}{0.965, 0.965, 0.965}\color{fgcolor}\begin{kframe}
\begin{alltt}
\hlstd{issp.df}\hlopt{$}\hlstd{Q1[}\hlnum{1}\hlopt{:}\hlnum{10}\hlstd{]}
\end{alltt}
\end{kframe}
\end{knitrout}
\vspace{-5mm}
\begin{knitrout}
\definecolor{shadecolor}{rgb}{0.965, 0.965, 0.965}\color{fgcolor}\begin{kframe}
\begin{verbatim}
 [1] disagree              strongly disagree    
 [3] disagree              <NA>                 
 [5] disagree              disagree             
 [7] strongly agree        neither agree nor dis
 [9] <NA>                  disagree             
5 Levels: strongly agree ... strongly disagree
\end{verbatim}
\end{kframe}
\end{knitrout}
\vspace{-2mm}
\begin{knitrout}
\definecolor{shadecolor}{rgb}{0.965, 0.965, 0.965}\color{fgcolor}\begin{kframe}
\begin{alltt}
\hlkwd{as.numeric}\hlstd{(issp.df}\hlopt{$}\hlstd{Q1[}\hlnum{1}\hlopt{:}\hlnum{10}\hlstd{])}
\end{alltt}
\end{kframe}
\end{knitrout}
\vspace{-5mm}
\begin{knitrout}
\definecolor{shadecolor}{rgb}{0.965, 0.965, 0.965}\color{fgcolor}\begin{kframe}
\begin{verbatim}
 [1]  4  5  4 NA  4  4  1  3 NA  4
\end{verbatim}
\end{kframe}
\end{knitrout}
\end{frame}

\begin{frame}[fragile]
\frametitle{\texttt{as.numeric()}}
Items \texttt{Q1} -- \texttt{Q4} are on the theme of {\em household gender roles}. Let's convert the values in each variable to a 5-point likert scale.
\begin{knitrout}
\definecolor{shadecolor}{rgb}{0.965, 0.965, 0.965}\color{fgcolor}\begin{kframe}
\begin{alltt}
\hlstd{Q1.lik} \hlkwb{<-} \hlkwd{as.numeric}\hlstd{(issp.df}\hlopt{$}\hlstd{Q1)}
\hlkwd{table}\hlstd{(Q1.lik)}
\end{alltt}
\begin{verbatim}
Q1.lik
  1   2   3   4   5 
 98 297 331 263  21 
\end{verbatim}
\begin{alltt}
\hlstd{Q2.lik} \hlkwb{<-} \hlkwd{as.numeric}\hlstd{(issp.df}\hlopt{$}\hlstd{Q2)}
\hlstd{Q3.lik} \hlkwb{<-} \hlkwd{as.numeric}\hlstd{(issp.df}\hlopt{$}\hlstd{Q3)}
\hlstd{Q4.lik} \hlkwb{<-} \hlkwd{as.numeric}\hlstd{(issp.df}\hlopt{$}\hlstd{Q4)}
\end{alltt}
\end{kframe}
\end{knitrout}
\end{frame}

\begin{frame}[fragile]
\frametitle{Items on household gender roles: Total score}
\begin{knitrout}
\definecolor{shadecolor}{rgb}{0.965, 0.965, 0.965}\color{fgcolor}\begin{kframe}
\begin{alltt}
\hlstd{total.lik} \hlkwb{<-} \hlstd{Q1.lik} \hlopt{+} \hlstd{Q2.lik} \hlopt{+} \hlstd{Q3.lik} \hlopt{+} \hlstd{Q4.lik}
\hlkwd{summary}\hlstd{(total.lik)}
\end{alltt}
\begin{verbatim}
   Min. 1st Qu.  Median    Mean 3rd Qu.    Max. 
   5.00   11.00   12.00   12.44   14.00   20.00 
   NA's 
     73 
\end{verbatim}
\end{kframe}
\end{knitrout}
\end{frame}

\begin{frame}[fragile]
\frametitle{Constructing a \texttt{data.frame}}
Let's create a new dataset from \texttt{Q1.lik} -- \texttt{Q4.lik} and \texttt{total.lik} called \texttt{likert.df}.
\begin{knitrout}
\definecolor{shadecolor}{rgb}{0.965, 0.965, 0.965}\color{fgcolor}\begin{kframe}
\begin{alltt}
\hlstd{likert.df} \hlkwb{<-} \hlkwd{data.frame}\hlstd{(Q1.lik, Q2.lik, Q3.lik, Q4.lik,}
                        \hlstd{total.lik)}
\hlkwd{head}\hlstd{(likert.df)}
\end{alltt}
\begin{verbatim}
  Q1.lik Q2.lik Q3.lik Q4.lik total.lik
1      4      2      3      2        11
2      5      3      4      2        14
3      4      5      5      2        16
4     NA      4      4      2        NA
5      4      3      4      2        13
6      4      4      4      2        14
\end{verbatim}
\end{kframe}
\end{knitrout}
\end{frame}

\begin{frame}[fragile]
\frametitle{Properties of \texttt{likert.df}}
What are its column names?

\vspace{-2mm}
\begin{knitrout}
\definecolor{shadecolor}{rgb}{0.965, 0.965, 0.965}\color{fgcolor}\begin{kframe}
\begin{alltt}
\hlkwd{names}\hlstd{(likert.df)}
\end{alltt}
\end{kframe}
\end{knitrout}
\vspace{-5mm}
\begin{knitrout}
\definecolor{shadecolor}{rgb}{0.965, 0.965, 0.965}\color{fgcolor}\begin{kframe}
\begin{verbatim}
[1] "Q1.lik"    "Q2.lik"    "Q3.lik"    "Q4.lik"   
[5] "total.lik"
\end{verbatim}
\end{kframe}
\end{knitrout}
\vspace{-2mm}
For data frames, \texttt{names()} yields {\em column} names. 
Let's now change them!\\
\vspace{-2mm}
\begin{knitrout}
\definecolor{shadecolor}{rgb}{0.965, 0.965, 0.965}\color{fgcolor}\begin{kframe}
\begin{alltt}
\hlcom{# Change only the last column's name to "Total"}
\hlkwd{names}\hlstd{(likert.df)[}\hlnum{5}\hlstd{]} \hlkwb{<-} \hlstr{"Total"}
\hlkwd{names}\hlstd{(likert.df)}
\end{alltt}
\end{kframe}
\end{knitrout}
\vspace{-5mm}
\begin{knitrout}
\definecolor{shadecolor}{rgb}{0.965, 0.965, 0.965}\color{fgcolor}\begin{kframe}
\begin{verbatim}
[1] "Q1.lik" "Q2.lik" "Q3.lik" "Q4.lik" "Total" 
\end{verbatim}
\end{kframe}
\end{knitrout}
\vspace{-2mm}
\begin{knitrout}
\definecolor{shadecolor}{rgb}{0.965, 0.965, 0.965}\color{fgcolor}\begin{kframe}
\begin{alltt}
\hlcom{# Change all columns names}
\hlkwd{names}\hlstd{(likert.df)} \hlkwb{<-} \hlkwd{c}\hlstd{(}\hlstr{"Q1"}\hlstd{,} \hlstr{"Q2"}\hlstd{,} \hlstr{"Q3"}\hlstd{,} \hlstr{"Q4"}\hlstd{,} \hlstr{"Total"}\hlstd{)}
\end{alltt}
\end{kframe}
\end{knitrout}
\vspace{-5mm}
\begin{knitrout}
\definecolor{shadecolor}{rgb}{0.965, 0.965, 0.965}\color{fgcolor}\begin{kframe}
\begin{verbatim}
[1] "Q1"    "Q2"    "Q3"    "Q4"    "Total"
\end{verbatim}
\end{kframe}
\end{knitrout}
\end{frame}

\begin{frame}[fragile]
\frametitle{Constructing a \texttt{data.frame}}
Let's create the column names we want during our construction of the data frame, i.e.
\begin{knitrout}
\definecolor{shadecolor}{rgb}{0.965, 0.965, 0.965}\color{fgcolor}\begin{kframe}
\begin{alltt}
\hlstd{likert.df} \hlkwb{<-} \hlkwd{data.frame}\hlstd{(}\hlkwc{Q1}\hlstd{=Q1.lik,} \hlkwc{Q2}\hlstd{=Q2.lik,}
                        \hlkwc{Q3}\hlstd{=Q3.lik,} \hlkwc{Q4}\hlstd{=Q4.lik,}
                        \hlkwc{Total}\hlstd{=total.lik)}
\hlkwd{head}\hlstd{(likert.df)}
\end{alltt}
\begin{verbatim}
  Q1 Q2 Q3 Q4 Total
1  4  2  3  2    11
2  5  3  4  2    14
3  4  5  5  2    16
4 NA  4  4  2    NA
5  4  3  4  2    13
6  4  4  4  2    14
\end{verbatim}
\end{kframe}
\end{knitrout}
\end{frame}

\begin{frame}[fragile]
\frametitle{Properties of \texttt{likert.df}}
\begin{knitrout}
\definecolor{shadecolor}{rgb}{0.965, 0.965, 0.965}\color{fgcolor}\begin{kframe}
\begin{alltt}
\hlkwd{str}\hlstd{(likert.df)}
\end{alltt}
\begin{verbatim}
'data.frame':	1047 obs. of  5 variables:
 $ Q1   : num  4 5 4 NA 4 4 1 3 NA 4 ...
 $ Q2   : num  2 3 5 4 3 4 3 5 NA 5 ...
 $ Q3   : num  3 4 5 4 4 4 4 5 NA 5 ...
 $ Q4   : num  2 2 2 2 2 2 2 1 NA 2 ...
 $ Total: num  11 14 16 NA 13 14 10 14 NA 16 ...
\end{verbatim}
\end{kframe}
\end{knitrout}
The \texttt{str{}} function tells us that \texttt{likert.df}:
\begin{enumerate}
  \item is a data frame, 
  \item comprises 1047 \texttt{obs.} (cases or rows) and 5 \texttt{variables} (columns), and 
  \item all 5 variables are \texttt{num}eric.
\end{enumerate}
\end{frame}

\begin{frame}[fragile]
\frametitle{\texttt{likert.df}: Column summary statistics}
I want to generate the summary statistics for all variables in \texttt{likert.df}.
\begin{knitrout}
\definecolor{shadecolor}{rgb}{0.965, 0.965, 0.965}\color{fgcolor}\begin{kframe}
\begin{alltt}
\hlkwd{summary}\hlstd{(likert.df}\hlopt{$}\hlstd{Q1)}
\end{alltt}
\begin{verbatim}
   Min. 1st Qu.  Median    Mean 3rd Qu.    Max.    NA's 
  1.000   2.000   3.000   2.814   4.000   5.000      37 
\end{verbatim}
\begin{alltt}
\hlkwd{summary}\hlstd{(likert.df}\hlopt{$}\hlstd{Q2)}
\end{alltt}
\begin{verbatim}
   Min. 1st Qu.  Median    Mean 3rd Qu.    Max.    NA's 
  1.000   3.000   4.000   3.531   4.000   5.000      27 
\end{verbatim}
\end{kframe}
\end{knitrout}
.\\
.\\
.\\
\end{frame}

\begin{frame}[fragile]
\frametitle{\texttt{for} loop to get column summary statistics}
\begin{knitrout}
\definecolor{shadecolor}{rgb}{0.965, 0.965, 0.965}\color{fgcolor}\begin{kframe}
\begin{alltt}
\hlkwa{for} \hlstd{(i} \hlkwa{in} \hlnum{1}\hlopt{:}\hlkwd{ncol}\hlstd{(likert.df))\{}
  \hlkwd{print}\hlstd{(}\hlkwd{summary}\hlstd{(likert.df[,i]))}
\hlstd{\}}
\end{alltt}
\begin{verbatim}
   Min. 1st Qu.  Median    Mean 3rd Qu.    Max.    NA's 
  1.000   2.000   3.000   2.814   4.000   5.000      37 
   Min. 1st Qu.  Median    Mean 3rd Qu.    Max.    NA's 
  1.000   3.000   4.000   3.531   4.000   5.000      27 
   Min. 1st Qu.  Median    Mean 3rd Qu.    Max.    NA's 
  1.000   3.000   4.000   3.716   4.000   5.000      26 
   Min. 1st Qu.  Median    Mean 3rd Qu.    Max.    NA's 
  1.000   2.000   2.000   2.307   3.000   5.000      19 
   Min. 1st Qu.  Median    Mean 3rd Qu.    Max.    NA's 
   5.00   11.00   12.00   12.44   14.00   20.00      73 
\end{verbatim}
\end{kframe}
\end{knitrout}
\end{frame}

\begin{frame}[fragile]
\frametitle{Smart way: \texttt(apply) to get column summary statistics}
\begin{knitrout}
\definecolor{shadecolor}{rgb}{0.965, 0.965, 0.965}\color{fgcolor}\begin{kframe}
\begin{alltt}
\hlkwd{apply}\hlstd{(likert.df,} \hlnum{2}\hlstd{, summary)}
\end{alltt}
\begin{verbatim}
            Q1     Q2     Q3     Q4 Total
Min.     1.000  1.000  1.000  1.000  5.00
1st Qu.  2.000  3.000  3.000  2.000 11.00
Median   3.000  4.000  4.000  2.000 12.00
Mean     2.814  3.531  3.716  2.307 12.44
3rd Qu.  4.000  4.000  4.000  3.000 14.00
Max.     5.000  5.000  5.000  5.000 20.00
NA's    37.000 27.000 26.000 19.000 73.00
\end{verbatim}
\end{kframe}
\end{knitrout}
\end{frame}  

\begin{frame}[fragile]
  \frametitle{\texttt{apply}}
\begin{verbatim}
apply(X, MARGIN, FUN, ...)
\end{verbatim}
  \begin{itemize}
  \item \texttt{X}: A data frame, e.g. \texttt{issp.df}.
  \item \texttt{MARGIN}: 1 indicates rows, 2 indicates columns.
  \item \texttt{FUN}: function, what do you want \texttt{R} to do with
    the rows or columns of the data frame
  \item \texttt{...}: optional arguments to \texttt{FUN}.
  \end{itemize}
Translation: Do something (\texttt{FUN}) to every row (or column) (\texttt{MARGIN}) of a data
frame (\texttt{X}). 
\end{frame}


\begin{frame}[fragile]
\frametitle{\texttt{apply()}}
Compute the mean and standard deviation of each column in \texttt{likert.df}, ignoring \texttt{NA}s.
\begin{knitrout}
\definecolor{shadecolor}{rgb}{0.965, 0.965, 0.965}\color{fgcolor}\begin{kframe}
\begin{alltt}
\hlkwd{apply}\hlstd{(likert.df,} \hlnum{2}\hlstd{, mean,} \hlkwc{na.rm} \hlstd{=} \hlnum{TRUE}\hlstd{)}
\end{alltt}
\begin{verbatim}
       Q1        Q2        Q3        Q4     Total 
 2.813861  3.531373  3.715965  2.307393 12.435318 
\end{verbatim}
\begin{alltt}
\hlkwd{apply}\hlstd{(likert.df,} \hlnum{2}\hlstd{, sd,} \hlkwc{na.rm} \hlstd{=} \hlnum{TRUE}\hlstd{)}
\end{alltt}
\begin{verbatim}
       Q1        Q2        Q3        Q4     Total 
0.9960307 1.1398725 1.0335688 0.8596006 2.3022901 
\end{verbatim}
\end{kframe}
\end{knitrout}
\end{frame}  

\begin{frame}[fragile]
\frametitle{\texttt{apply()} using self-defined \texttt{R} function}
Functions used in \texttt{apply()} can be self-defined.
\begin{knitrout}
\definecolor{shadecolor}{rgb}{0.965, 0.965, 0.965}\color{fgcolor}\begin{kframe}
\begin{alltt}
\hlstd{na.check} \hlkwb{<-} \hlkwa{function}\hlstd{(}\hlkwc{someinput}\hlstd{)\{}
 \hlstd{test.na} \hlkwb{<-} \hlkwd{is.na}\hlstd{(someinput)}
 \hlkwd{sum}\hlstd{(test.na)}
\hlstd{\}}
\end{alltt}
\end{kframe}
\end{knitrout}
Take an educated guess at what \texttt{na.check()} does.
\end{frame}  

\begin{frame}[fragile]
\frametitle{\texttt{apply()} using a self-defined \texttt{R} function}
Let's look at what each row of \texttt{na.check()} does.
\begin{knitrout}
\definecolor{shadecolor}{rgb}{0.965, 0.965, 0.965}\color{fgcolor}\begin{kframe}
\begin{alltt}
\hlstd{test1} \hlkwb{<-} \hlstd{issp.df}\hlopt{$}\hlstd{Age[}\hlnum{1}\hlopt{:}\hlnum{10}\hlstd{]}
\hlstd{test1}
\end{alltt}
\begin{verbatim}
 [1] 56 45 38 33 37 27 43 24 NA 22
\end{verbatim}
\end{kframe}
\end{knitrout}
\begin{knitrout}
\definecolor{shadecolor}{rgb}{0.965, 0.965, 0.965}\color{fgcolor}\begin{kframe}
\begin{alltt}
\hlstd{test.na} \hlkwb{<-} \hlkwd{is.na}\hlstd{(test1)}
\hlstd{test.na}
\end{alltt}
\end{kframe}
\end{knitrout}
\begin{knitrout}
\definecolor{shadecolor}{rgb}{0.965, 0.965, 0.965}\color{fgcolor}\begin{kframe}
\begin{verbatim}
 [1] FALSE FALSE FALSE FALSE FALSE FALSE FALSE
 [8] FALSE  TRUE FALSE
\end{verbatim}
\begin{alltt}
\hlkwd{sum}\hlstd{(test.na)}
\end{alltt}
\begin{verbatim}
[1] 1
\end{verbatim}
\end{kframe}
\end{knitrout}
\end{frame} 

\begin{frame}[fragile]
\frametitle{\texttt{apply()} using a self-defined \texttt{R} function}
Let's now use \texttt{na.check()} in \texttt{apply()}.
\begin{knitrout}
\definecolor{shadecolor}{rgb}{0.965, 0.965, 0.965}\color{fgcolor}\begin{kframe}
\begin{alltt}
\hlkwd{apply}\hlstd{(likert.df,} \hlnum{2}\hlstd{, na.check)}
\end{alltt}
\begin{verbatim}
   Q1    Q2    Q3    Q4 Total 
   37    27    26    19    73 
\end{verbatim}
\end{kframe}
\end{knitrout}
\vspace{4mm}
Attention programmers!
\begin{knitrout}
\definecolor{shadecolor}{rgb}{0.965, 0.965, 0.965}\color{fgcolor}\begin{kframe}
\begin{alltt}
\hlkwd{apply}\hlstd{(likert.df,} \hlnum{2}\hlstd{,} \hlkwa{function}\hlstd{(}\hlkwc{x}\hlstd{)} \hlkwd{sum}\hlstd{(}\hlkwd{is.na}\hlstd{(x)))}
\end{alltt}
\begin{verbatim}
   Q1    Q2    Q3    Q4 Total 
   37    27    26    19    73 
\end{verbatim}
\end{kframe}
\end{knitrout}
\end{frame} 


\begin{frame}[fragile]
\frametitle{A slightly more complicated function}
\begin{knitrout}
\definecolor{shadecolor}{rgb}{0.965, 0.965, 0.965}\color{fgcolor}\begin{kframe}
\begin{alltt}
\hlstd{mytab} \hlkwb{<-} \hlkwa{function}\hlstd{(}\hlkwc{someinput}\hlstd{)\{}
 \hlstd{n} \hlkwb{<-} \hlkwd{length}\hlstd{(someinput)}
 \hlstd{n.missing} \hlkwb{<-} \hlkwd{na.check}\hlstd{(someinput)}
 \hlstd{n.complete} \hlkwb{<-} \hlstd{n} \hlopt{-} \hlstd{n.missing}
 \hlstd{mymean} \hlkwb{<-} \hlkwd{round}\hlstd{(}\hlkwd{mean}\hlstd{(someinput,} \hlkwc{na.rm} \hlstd{=} \hlnum{TRUE}\hlstd{),} \hlnum{2}\hlstd{)}
 \hlstd{mysd} \hlkwb{<-} \hlkwd{round}\hlstd{(}\hlkwd{sd}\hlstd{(someinput,} \hlkwc{na.rm} \hlstd{=} \hlnum{TRUE}\hlstd{),} \hlnum{2}\hlstd{)}
 \hlstd{mystder} \hlkwb{<-} \hlkwd{round}\hlstd{(mysd}\hlopt{/}\hlkwd{sqrt}\hlstd{(n.complete),} \hlnum{2}\hlstd{)}
 \hlstd{Lower.CI} \hlkwb{<-} \hlkwd{round}\hlstd{(mymean} \hlopt{-} \hlnum{1.96}\hlopt{*}\hlstd{mystder,} \hlnum{2}\hlstd{)}
 \hlstd{Upper.CI} \hlkwb{<-} \hlkwd{round}\hlstd{(mymean} \hlopt{+} \hlnum{1.96}\hlopt{*}\hlstd{mystder,} \hlnum{2}\hlstd{)}
 \hlkwd{c}\hlstd{(}\hlkwc{Complete.obs} \hlstd{= n.complete,} \hlkwc{Missing.obs} \hlstd{= n.missing,}
   \hlkwc{Mean} \hlstd{= mymean,} \hlkwc{Std.Error} \hlstd{= mystder,}
   \hlkwc{Lower.CI} \hlstd{= Lower.CI,} \hlkwc{Upper.CI} \hlstd{= Upper.CI)}
\hlstd{\}}
\end{alltt}
\end{kframe}
\end{knitrout}
Take a \emph{more} educated guess at what \texttt{mytab()} does?
\end{frame}  

\begin{frame}[fragile]
\frametitle{A slightly more complicated function}
\begin{itemize}
  \item For the \texttt{R} novice, \texttt{mytab()} is possibly terrifying!
  \item We too were \texttt{R} novices once!
  \item Our advice on understanding what an \texttt{R} function does?\\[1em]
  \item[] \begin{center}
  ``Use a data set for input into the function and work through it one line of code at a time.''\\[1em]
  \end{center}
  \item We ``experts'' still do this!
\end{itemize}  
\end{frame}  

\begin{frame}[fragile]
\frametitle{\texttt{mytab()}}
\begin{knitrout}
\definecolor{shadecolor}{rgb}{0.965, 0.965, 0.965}\color{fgcolor}\begin{kframe}
\begin{alltt}
\hlkwd{apply}\hlstd{(likert.df,} \hlnum{2}\hlstd{, mytab)}
\end{alltt}
\begin{verbatim}
                  Q1      Q2      Q3      Q4  Total
Complete.obs 1010.00 1020.00 1021.00 1028.00 974.00
Missing.obs    37.00   27.00   26.00   19.00  73.00
Mean            2.81    3.53    3.72    2.31  12.44
Std.Error       0.03    0.04    0.03    0.03   0.07
Lower.CI        2.75    3.45    3.66    2.25  12.30
Upper.CI        2.87    3.61    3.78    2.37  12.58
\end{verbatim}
\end{kframe}
\end{knitrout}
\end{frame} 


\begin{frame}[fragile]
\frametitle{More descriptive stats}
Calculate the mean total score for male and female respondents.
\begin{knitrout}
\definecolor{shadecolor}{rgb}{0.965, 0.965, 0.965}\color{fgcolor}\begin{kframe}
\begin{alltt}
\hlstd{issp.df}\hlopt{$}\hlstd{total.lik} \hlkwb{<-} \hlstd{likert.df}\hlopt{$}\hlstd{Total}
\hlkwd{with}\hlstd{(issp.df,} \hlkwd{mean}\hlstd{(total.lik[Gender} \hlopt{==} \hlstr{"Male"}\hlstd{],}
                   \hlkwc{na.rm} \hlstd{=} \hlnum{TRUE}\hlstd{))}
\end{alltt}
\begin{verbatim}
[1] 12.06436
\end{verbatim}
\begin{alltt}
\hlkwd{with}\hlstd{(issp.df,} \hlkwd{mean}\hlstd{(total.lik[Gender} \hlopt{==} \hlstr{"Female"}\hlstd{],}
                   \hlkwc{na.rm} \hlstd{=} \hlnum{TRUE}\hlstd{))}
\end{alltt}
\begin{verbatim}
[1] 12.71067
\end{verbatim}
\end{kframe}
\end{knitrout}
How about calculating the mean total score for each income group (10 levels)?
\end{frame} 

\begin{frame}[fragile]
\frametitle{Smart way}
\begin{knitrout}
\definecolor{shadecolor}{rgb}{0.965, 0.965, 0.965}\color{fgcolor}\begin{kframe}
\begin{alltt}
\hlstd{issp.df}\hlopt{$}\hlstd{Income} \hlkwb{<-} \hlkwd{ifelse}\hlstd{(issp.df}\hlopt{$}\hlstd{Income}
                         \hlopt{==} \hlstr{"NAV; NAP No own income"}\hlstd{,}
                         \hlnum{NA}\hlstd{, issp.df}\hlopt{$}\hlstd{Income)}
\hlkwd{with}\hlstd{(issp.df,} \hlkwd{tapply}\hlstd{(total.lik, Income, mean,} \hlkwc{na.rm} \hlstd{=} \hlnum{TRUE}\hlstd{))}
\end{alltt}
\begin{verbatim}
$10000 or less  $10001-$15000  $15001-$20000 
      12.11261       12.19403       12.57333 
 $20001-$25000  $25001-$30000  $30001-$40000 
      12.36697       12.84071       12.79688 
 $40001-$50000  $50001-$70000 $70001-$100000 
      12.94444       12.52941       12.65000 
\end{verbatim}
\end{kframe}
\end{knitrout}
\end{frame} 

\begin{frame}[fragile]
\frametitle{\texttt{tapply()}}
\begin{knitrout}
\definecolor{shadecolor}{rgb}{0.965, 0.965, 0.965}\color{fgcolor}\begin{kframe}
\begin{alltt}
\hlkwd{with}\hlstd{(issp.df,} \hlkwd{tapply}\hlstd{(total.lik, Income, mean,} \hlkwc{na.rm} \hlstd{=} \hlnum{TRUE}\hlstd{))}
\end{alltt}
\end{kframe}
\end{knitrout}
\begin{verbatim}
tapply(X, INDEX, FUN, ...)
\end{verbatim}
Translation: Apply function \texttt{FUN} to \texttt{X} for each level in the grouping factor \texttt{INDEX}
\end{frame}    

\begin{frame}[fragile]
\frametitle{\texttt{tapply()}}
\begin{knitrout}
\definecolor{shadecolor}{rgb}{0.965, 0.965, 0.965}\color{fgcolor}\begin{kframe}
\begin{alltt}
\hlkwd{with}\hlstd{(issp.df,} \hlkwd{tapply}\hlstd{(total.lik, Gender, mytab))}
\end{alltt}
\begin{verbatim}
$Female
Complete.obs  Missing.obs         Mean    Std.Error 
      553.00        54.00        12.71         0.10 
    Lower.CI     Upper.CI 
       12.51        12.91 

$Male
Complete.obs  Missing.obs         Mean    Std.Error 
      404.00        14.00        12.06         0.11 
    Lower.CI     Upper.CI 
       11.84        12.28 
\end{verbatim}
\end{kframe}
\end{knitrout}
\end{frame} 

\begin{frame}[fragile]
  \frametitle{Summary}
  \begin{itemize}
  \item Recoding
  \item Summary Stats
  \item \texttt{apply()}
  \item \texttt{tapply()}
  \end{itemize}
\end{frame}

\end{document}     
