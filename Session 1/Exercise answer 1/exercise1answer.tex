

\documentclass[12pt,a4paper]{article}\usepackage[]{graphicx}\usepackage[]{color}
%% maxwidth is the original width if it is less than linewidth
%% otherwise use linewidth (to make sure the graphics do not exceed the margin)
\makeatletter
\def\maxwidth{ %
  \ifdim\Gin@nat@width>\linewidth
    \linewidth
  \else
    \Gin@nat@width
  \fi
}
\makeatother

\definecolor{fgcolor}{rgb}{0.345, 0.345, 0.345}
\newcommand{\hlnum}[1]{\textcolor[rgb]{0.686,0.059,0.569}{#1}}%
\newcommand{\hlstr}[1]{\textcolor[rgb]{0.192,0.494,0.8}{#1}}%
\newcommand{\hlcom}[1]{\textcolor[rgb]{0.678,0.584,0.686}{\textit{#1}}}%
\newcommand{\hlopt}[1]{\textcolor[rgb]{0,0,0}{#1}}%
\newcommand{\hlstd}[1]{\textcolor[rgb]{0.345,0.345,0.345}{#1}}%
\newcommand{\hlkwa}[1]{\textcolor[rgb]{0.161,0.373,0.58}{\textbf{#1}}}%
\newcommand{\hlkwb}[1]{\textcolor[rgb]{0.69,0.353,0.396}{#1}}%
\newcommand{\hlkwc}[1]{\textcolor[rgb]{0.333,0.667,0.333}{#1}}%
\newcommand{\hlkwd}[1]{\textcolor[rgb]{0.737,0.353,0.396}{\textbf{#1}}}%
\let\hlipl\hlkwb

\usepackage{framed}
\makeatletter
\newenvironment{kframe}{%
 \def\at@end@of@kframe{}%
 \ifinner\ifhmode%
  \def\at@end@of@kframe{\end{minipage}}%
  \begin{minipage}{\columnwidth}%
 \fi\fi%
 \def\FrameCommand##1{\hskip\@totalleftmargin \hskip-\fboxsep
 \colorbox{shadecolor}{##1}\hskip-\fboxsep
     % There is no \\@totalrightmargin, so:
     \hskip-\linewidth \hskip-\@totalleftmargin \hskip\columnwidth}%
 \MakeFramed {\advance\hsize-\width
   \@totalleftmargin\z@ \linewidth\hsize
   \@setminipage}}%
 {\par\unskip\endMakeFramed%
 \at@end@of@kframe}
\makeatother

\definecolor{shadecolor}{rgb}{.97, .97, .97}
\definecolor{messagecolor}{rgb}{0, 0, 0}
\definecolor{warningcolor}{rgb}{1, 0, 1}
\definecolor{errorcolor}{rgb}{1, 0, 0}
\newenvironment{knitrout}{}{} % an empty environment to be redefined in TeX

\usepackage{alltt}
\usepackage{amsmath}
\usepackage{enumerate}
\usepackage[cm]{fullpage}
\IfFileExists{upquote.sty}{\usepackage{upquote}}{}
\begin{document}
\setlength\parindent{0cm}
\title{\Large{\textbf{Introduction to \texttt{R}}}\\
\textit{Answers to Session 1 exercises}}
\author{Statistical Consulting Centre}
\date{1 March, 2017}
\maketitle

\section{Using \textbf{R} as a calculator}
\label{sec:use-r-as}

\begin{enumerate}[(i)]

\item \label{itm:calculator} Find the values of:
  \begin{enumerate}

  \item $1+4$
\begin{knitrout}
\definecolor{shadecolor}{rgb}{0.969, 0.969, 0.969}\color{fgcolor}\begin{kframe}
\begin{alltt}
\hlnum{1}\hlopt{+}\hlnum{4}
\end{alltt}
\begin{verbatim}
[1] 5
\end{verbatim}
\end{kframe}
\end{knitrout}
  \item $2^3 + \frac{4}{\sqrt{34}}$
\begin{knitrout}
\definecolor{shadecolor}{rgb}{0.969, 0.969, 0.969}\color{fgcolor}\begin{kframe}
\begin{alltt}
\hlnum{2}\hlopt{^}\hlnum{3} \hlopt{+} \hlnum{4}\hlopt{/}\hlkwd{sqrt}\hlstd{(}\hlnum{34}\hlstd{)}
\end{alltt}
\begin{verbatim}
[1] 8.685994
\end{verbatim}
\end{kframe}
\end{knitrout}
  \item $\log{30}$
\begin{knitrout}
\definecolor{shadecolor}{rgb}{0.969, 0.969, 0.969}\color{fgcolor}\begin{kframe}
\begin{alltt}
\hlkwd{log}\hlstd{(}\hlnum{30}\hlstd{)}
\end{alltt}
\begin{verbatim}
[1] 3.401197
\end{verbatim}
\end{kframe}
\end{knitrout}

  \item $\log_{10}30$
\begin{knitrout}
\definecolor{shadecolor}{rgb}{0.969, 0.969, 0.969}\color{fgcolor}\begin{kframe}
\begin{alltt}
\hlkwd{log10}\hlstd{(}\hlnum{30}\hlstd{)}
\end{alltt}
\begin{verbatim}
[1] 1.477121
\end{verbatim}
\end{kframe}
\end{knitrout}

  \item $|-2|$ \hspace{0.2cm}(Hint: $|x|$ denotes the \emph{absolute
      value} of $x$. Search on Google if you're unsure.)
\begin{knitrout}
\definecolor{shadecolor}{rgb}{0.969, 0.969, 0.969}\color{fgcolor}\begin{kframe}
\begin{alltt}
\hlkwd{abs}\hlstd{(}\hlopt{-}\hlnum{2}\hlstd{)}
\end{alltt}
\begin{verbatim}
[1] 2
\end{verbatim}
\end{kframe}
\end{knitrout}

  \end{enumerate}
\item Now open Rstudio, then open a new \texttt{R} script by clicking \texttt{File}
  $\rightarrow$ \texttt{New} $\rightarrow$ \texttt{R} script.
\item Save this script by clicking \texttt{File} $\rightarrow$
  \texttt{Save As...}.
\item Select a directory/location and save the script. Note: the saved
  script should have \texttt{.r} as extension. For example, if you
  call your file \texttt{exercise one}, then you should save it as
  \texttt{exercise one.r}
\item Copy and paste the code you typed (\emph{not the output, not the
  $>$ symbol, just the code you typed}) at the console for {\bf
    \ref{sec:use-r-as}}(\ref{itm:calculator}) into the \texttt{R}
  script opened in Rstudio.
\item Submit your entire script at once to the \texttt{R}
  Console by highlighting all codes and pressing Ctrl $+$ R.
\item From now on, type all of your code in your \texttt{R} script and
  submit it to the \texttt{R} Console using Ctrl $+$ R.

\end{enumerate}

\section{Reading data into \texttt{R}}
\label{sec:read}

\begin{enumerate}[(i)]

\item We are going to use Leisure Time and Sports Questionnaire done
  by ISSP at 2007 for our exercises. Again we only take a small
  proportion of the survey (\texttt{sports.csv}). Please see the
  questionnaire provided.
\item Read the data into \texttt{R}, saving it in an object named
  \texttt{sports.df}.
\begin{knitrout}
\definecolor{shadecolor}{rgb}{0.969, 0.969, 0.969}\color{fgcolor}\begin{kframe}
\begin{alltt}
\hlstd{sports.df} \hlkwb{<-} \hlkwd{read.csv}\hlstd{(}\hlstr{"location of your folder/sports.csv"}\hlstd{,}
                      \hlkwc{stringsAsFactors} \hlstd{=} \hlnum{FALSE}\hlstd{)}
\end{alltt}
\end{kframe}
\end{knitrout}

\item Use \texttt{dim()} and \texttt{head()} to look at some of the
  properties of the dataset you have just read into
  \texttt{R}. \emph{Always} perform these two important checks of your
  data to ensure what you have read into \texttt{R} is as it should
  be.
\begin{knitrout}
\definecolor{shadecolor}{rgb}{0.969, 0.969, 0.969}\color{fgcolor}\begin{kframe}
\begin{alltt}
\hlkwd{dim}\hlstd{(sports.df)}
\end{alltt}
\begin{verbatim}
[1] 996  11
\end{verbatim}
\begin{alltt}
\hlkwd{head}\hlstd{(sports.df)}
\end{alltt}
\begin{verbatim}
  id                                q1a                                q1b
1  2 Several times a year or less often Several times a year or less often
2  4 Several times a year or less often                       Can´t choose
3  8                       Can´t choose                       Can´t choose
4  9 Several times a year or less often Several times a year or less often
5 10 Several times a year or less often Several times a year or less often
6 12                       Can´t choose Several times a year or less often
                                 q1c                                q1d
1 Several times a year or less often               Several times a week
2              Several times a month Several times a year or less often
3                               <NA>              Several times a month
4               Several times a week              Several times a month
5               Several times a week              Several times a month
6               Several times a week                              Daily
                                 q1e age gender
1                              Daily  31   Male
2 Several times a year or less often  50   Male
3                              Daily  52 Female
4                       Can´t choose  67 Female
5               Several times a week  71 Female
6                              Daily  33 Female
                                                     partner
1 NAP (married and living w legal spouse, Code 1 in MARITAL)
2 NAP (married and living w legal spouse, Code 1 in MARITAL)
3 NAP (married and living w legal spouse, Code 1 in MARITAL)
4                                                         No
5 NAP (married and living w legal spouse, Code 1 in MARITAL)
6 NAP (married and living w legal spouse, Code 1 in MARITAL)
              ethnicity          income
1 Europe,White/European 50 000$-70 000$
2 Europe,White/European 15 000$-20 000$
3 Europe,White/European 30 000$-40 000$
4 Europe,White/European 15 000$-20 000$
5 Europe,White/European 15 000$-20 000$
6 Europe,White/European      > 100 000$
\end{verbatim}
\end{kframe}
\end{knitrout}

\item Calculate the mean and standard deviation of age.
\begin{knitrout}
\definecolor{shadecolor}{rgb}{0.969, 0.969, 0.969}\color{fgcolor}\begin{kframe}
\begin{alltt}
\hlkwd{mean}\hlstd{(sports.df}\hlopt{$}\hlstd{age,} \hlkwc{na.rm} \hlstd{=} \hlnum{TRUE}\hlstd{)}
\end{alltt}
\begin{verbatim}
[1] 51.82831
\end{verbatim}
\begin{alltt}
\hlkwd{sd}\hlstd{(sports.df}\hlopt{$}\hlstd{age,} \hlkwc{na.rm} \hlstd{=} \hlnum{TRUE}\hlstd{)}
\end{alltt}
\begin{verbatim}
[1] 17.28576
\end{verbatim}
\end{kframe}
\end{knitrout}

\item Check the frequency of gender.
\begin{knitrout}
\definecolor{shadecolor}{rgb}{0.969, 0.969, 0.969}\color{fgcolor}\begin{kframe}
\begin{alltt}
\hlkwd{table}\hlstd{(sports.df}\hlopt{$}\hlstd{gender)}
\end{alltt}
\begin{verbatim}

Female   Male 
   535    461 
\end{verbatim}
\end{kframe}
\end{knitrout}
\item \label{itm:2freq} Produce a two-way frequency table between
  ethnicity and age.
\begin{knitrout}
\definecolor{shadecolor}{rgb}{0.969, 0.969, 0.969}\color{fgcolor}\begin{kframe}
\begin{alltt}
\hlkwd{with}\hlstd{(sports.df,} \hlkwd{table}\hlstd{(ethnicity, gender))}
\end{alltt}
\begin{verbatim}
                                  gender
ethnicity                          Female Male
  China,Cantonese,Hakka,Mandarin        8   11
  Europe,White/European               430  387
  India,Hindi,Urdu,Gujarati,Tamil       7    1
  Maori+New Zealand                    59   31
  NA, dont know                         5    6
  Other,mixed origin                   12   14
  PACIFIC,Polynesian,Chamorro/Guam     14   11
\end{verbatim}
\end{kframe}
\end{knitrout}

\item Turn the frequency table in \ref{sec:read}(\ref{itm:2freq})
into column proportions, keep only 1 decimal place.
\begin{knitrout}
\definecolor{shadecolor}{rgb}{0.969, 0.969, 0.969}\color{fgcolor}\begin{kframe}
\begin{alltt}
\hlstd{gen.eth.tab} \hlkwb{<-} \hlkwd{with}\hlstd{(sports.df,} \hlkwd{table}\hlstd{(ethnicity, gender))}
\hlkwd{round}\hlstd{(}\hlkwd{prop.table}\hlstd{(gen.eth.tab,} \hlnum{2}\hlstd{)}\hlopt{*}\hlnum{100}\hlstd{,} \hlnum{1}\hlstd{)}
\end{alltt}
\begin{verbatim}
                                  gender
ethnicity                          Female Male
  China,Cantonese,Hakka,Mandarin      1.5  2.4
  Europe,White/European              80.4 83.9
  India,Hindi,Urdu,Gujarati,Tamil     1.3  0.2
  Maori+New Zealand                  11.0  6.7
  NA, dont know                       0.9  1.3
  Other,mixed origin                  2.2  3.0
  PACIFIC,Polynesian,Chamorro/Guam    2.6  2.4
\end{verbatim}
\end{kframe}
\end{knitrout}

\item Now turn the frequency table in \ref{sec:read}(\ref{itm:2freq})
  into row proportions, keep only 1 decimal place.
\begin{knitrout}
\definecolor{shadecolor}{rgb}{0.969, 0.969, 0.969}\color{fgcolor}\begin{kframe}
\begin{alltt}
\hlkwd{round}\hlstd{(}\hlkwd{prop.table}\hlstd{(gen.eth.tab,} \hlnum{1}\hlstd{)}\hlopt{*}\hlnum{100}\hlstd{,} \hlnum{1}\hlstd{)}
\end{alltt}
\begin{verbatim}
                                  gender
ethnicity                          Female Male
  China,Cantonese,Hakka,Mandarin     42.1 57.9
  Europe,White/European              52.6 47.4
  India,Hindi,Urdu,Gujarati,Tamil    87.5 12.5
  Maori+New Zealand                  65.6 34.4
  NA, dont know                      45.5 54.5
  Other,mixed origin                 46.2 53.8
  PACIFIC,Polynesian,Chamorro/Guam   56.0 44.0
\end{verbatim}
\end{kframe}
\end{knitrout}

\end{enumerate}
\end{document}
